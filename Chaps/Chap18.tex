%!TEX root = ../CallenThermo.tex
\chapter{量子流体}\label{chap18}
\section{量子粒子:一个“预备 Fermi 气体模型”}\label{sec18.1}
此刻我们急切地想将巨正则系综理论运用在理想气体之上,这不是为了得到什么新的结果,而是希望能比较不同理论在解析计算上的便利与强度。值得注意的是,对于经典理想气体模型,巨正则系综理论显得非常{\it 不}友善!非广延性灾难不仅困扰着正则系综理论下的计算,对于巨正则系综理论来说尤甚%
\footnote{这个困难来源于巨正则系综理论关注与{\it 轨道态},而不是粒子。事实上不存在一种自然的方式``在假定全部粒子都有标签的情况下''(这样就能用一个$\tilde N!$因子来修正)来数态。}%
。

正如物理中所常见的,公式总能指向事实%
\mpar{译注:我们提请读者注意区分,究竟是根据物理事实总结规律再将其以最顺畅的数学形式表达,还是从部分归纳的规律出发,以数学上的简明性去对未知事物进行猜测与判断;前者是标准的科学步骤,不会有错,而后者是一种 taste ,可能好可能坏。}%
。理论体系所面临的困难,是该{\it 模型}的非物理性的一个信号——世界上没有经典粒子!只有 fermion 和 boson 这两类量子的粒子。对于它们,巨正则系综理论会变得相当简单!

fermion 对应于经典物理中的物质粒子。电子、质子、中微子,以及一大堆只有学高能的人才懂的粒子都属于 fermion 。十九世纪的``物质的不可入性原理'' (law of impenetrability of matter)被量子波函数的反对称条件所替代%
\footnote{在交换两个fermion时波函数必须反号,这会在fermion之间插入一个节点,并使得两个(处于同样自旋态)的fermion不能同时占据同样的空间位置。}%
。这个条件意味着(也是我们想要的){\it 一个轨道态上只能有一个 ferimion 占据。}

boson 对应于经典物理中的``波动''%
\mpar{译注:这几段对量子力学的介绍都是相当粗糙的定性描述,许多陈述都只在特定条件下适用,非常容易引起误解。读者若想准确认识这些东西,需要学习量子场论的相关知识。}%
。光子,光量子,是典型的 boson。正如经典中波可以任意叠加,{\it 同一个轨道态上可以有任意多个 boson 占据。}此外,有些 boson 的静质量是零——这样的 boson 像经典的波一样,可以被任意的产生和湮灭。在量子力学的表述下,高温物体的电磁辐射被理解成光子的产生和吸收。

基本粒子具有内禀的角动量,称为``自旋'' (spin)。其(不变的)大小是$\hbar/2$的整数倍;具有奇数倍$\hbar/2$的粒子是 fermion ,具有偶数倍$\hbar/2$的粒子是 boson 。

内禀角动量的朝向同样是量子化的。对于``自旋$\frac{1}{2}$''(角动量$=\hbar/2$)的 fermion ,其角动量(沿任意给定轴)只能有两个朝向。这两个朝向被记做{\it 上}和{\it 下},或者用``磁量子数'' (magnetic quantum number)$m_s$来标记为$m_s=\frac{1}{2}$和$m_s=-\frac{1}{2}$。

最后,量子粒子的一个轨道态需要{\it 同时}用其空间波函数的量子数和自旋取向的磁量子数$m_s$来标记。对于立方盒子中的粒子,三个空间量子数是波矢$\mathbf k$的三个分量(回忆\eqref{equ16.43}式)%
\mpar{译注:原文为16.37式,应为引用错误。}%
,因此一个轨道态就可以用$\mathbf k$和$m_s$来标记。

作为将巨正则系综理论应用于 Fermi 和 Bose 理想气体的准备工作,我们值得先考虑一个能把物理展示清楚的简单模型。这个模型只有三个能级,因而对所用能级求和是简单明了的。除了这个简化之外,对于该模型的分析与后续章节对于量子气体的分析一步一步都是严格对应的,故我们将其称为{\it 预备气体模型~(pre-gas model)}。

我们先考虑自旋$1/2$的预备 fermion 气体模型。在这个模型系统中,只有三个容许的空间轨道态;粒子在这些空间轨道态上的能量分别是$\varepsilon_1,\varepsilon_2$和$\varepsilon_3$。其与一个热库以及自旋$1/2$的fermi粒子库接触。库的温度是$T$,molar Gibbs 势为$\mu$(对于fermion系统,其称为{\it Fermi 能级~(Fermi level)})。

每个空间轨道态对应于两个轨道态,一个自旋向上一个自旋向下。总共就有六个轨道态,用$(n,m_s)$标记,其中$n=1,2,3$,$m=-\frac{1}{2},+\frac{1}{2}$。

关于这六个轨道态的巨正则配分求和因子是
\begin{equation}
\mathcal Z = z_{1,-1/2}z_{1,1/2}z_{2,-1/2}z_{2,1/2}z_{3,-1/2}z_{3,1/2},
\label{equ18.1}
\end{equation}
每个轨道态的配分求和有两项,对应于这个态被占据或未被占据。在没有磁场的情况下
\begin{equation}
z_{n,m_s}=1+\mathrm e^{-\beta(\varepsilon_n-\mu)}.
\label{equ18.2}
\end{equation}

另一种操作是,把具有相同$n$但是$m_s=\pm\frac{1}{2}$的两个轨道态放在一起
\begin{equation}
z_{n,1/2}z_{n,-1/2}=\left[1+\mathrm e^{-\beta(\varepsilon_n-\mu)}\right]^2=1+2\mathrm e^{-\beta(\varepsilon_n-\mu)}+\mathrm e^{-2\beta(\varepsilon_n-\mu)}.
\label{equ18.3}
\end{equation}
这个乘积可以理解成给定的$n$上有{\it 四}个态:空态,两个单占据态以及一个双占据态。

轨道态$(n,m_s)$为空的概率是$1/z_{n,m}$,其被占据的概率为
\begin{equation}
f_{n,m}=\frac{\mathrm e^{-\beta(\varepsilon_n-\mu)}}{z_{n,m}}=\frac{1}{e^{\beta(\varepsilon_n-\mu)}+1}.
\label{equ18.4}
\end{equation}

从\ref{equ18.1}到\ref{equ18.3}式可以直接导出基本方程
\begin{equation}
\mathrm e^{-\beta\Psi}=\mathcal Z = \left[1+\mathrm e^{-\beta(\varepsilon_1-\mu)}\right]^2\left[1+\mathrm e^{-\beta(\varepsilon_2-\mu)}\right]^2\left[1+\mathrm e^{-\beta(\varepsilon_3-\mu)}\right]^2.
\label{equ18.5}
\end{equation}

通过求导可以求得平均粒子数($\tilde N=-\partial\Psi/\partial\mu$)。另一条路子是对六个轨道态的占据概率求和
\begin{equation}
\tilde N = \sum\limits_{n,m}f_{n,m} = \frac{2}{e^{\beta(\varepsilon_1-\mu)}+1}+\frac{2}{e^{\beta(\varepsilon_2-\mu)}+1}+\frac{2}{e^{\beta(\varepsilon_3-\mu)}+1}.
\label{equ18.6}
\end{equation}

系统的熵可以通过对基本方程求导得到($S=-\partial\Psi/\partial T$)。另一条路是由占据概率去算(习题18.1-1)。

能量可以用热力学的办法求导:$U=(\partial\beta\Psi/\partial\beta)_{\beta\mu}$(\eqref{equ17.27}式)。另外,从$f_{n,m}$的概率诠释出发也能得到
\begin{equation}
U = \sum\limits_{n,m}f_{n,m} = \frac{2\varepsilon_1}{e^{\beta(\varepsilon_1-\mu)}+1}+\frac{2\varepsilon_2}{e^{\beta(\varepsilon_2-\mu)}+1}+\frac{2\varepsilon_3}{e^{\beta(\varepsilon_3-\mu)}+1}.
\label{equ18.7}
\end{equation}

如果这个系统确实是和$T$和$\mu$的热库接触,那么这个结果是很方便的。但是我们可能会对那些包围在封闭的墙体内的,粒子数$\tilde N$而不是$\mu$守恒的物理系统感兴趣。幸好{\it 基本方程是热力学系统的内在属性,与边界条件无关},因而之前的结果依然有效。尽管如此,Fermi 能级$\mu$是一个未知量。反而$\mu$的值随着温度变化而变化以维持$\tilde N$是一个常数,相应的变化由\eqref{equ18.6}给出。

不幸的是,\eqref{equ18.6}式并不能给出一个$\mu$关于$T$和$\tilde N$的显式表达式。当然正如我们接下来所能看到的,可以在特定温度区域中作数值解或者级数解。但先从图像上来重新考虑一下之前的分析会是有帮助的。

\begin{figure}
\includegraphics[width=\textwidth]{Pictures/fig18.1.png}
\figcaption{fermion 在温度$T$下占据一个能量为$\varepsilon$的轨道态的概率。}
\label{fig18.1}
\end{figure}

能量为$\varepsilon$的轨道态被占据的概率$f$(由\eqref{equ18.4}式给出)绘制于图\ref{fig18.1}中。这个占据概率当然比现在所讨论的模型要一般得多。它对于任何 fermion 的轨道态都适用。在零温极限下,全部能量$\varepsilon<\mu$的态都被占据而全部能量$\varepsilon>\mu$的态都是空的。当温度升高,能量略低于$\mu$的态上的布居开始逐渐下降,而能量略高于$\mu$的态上的布居开始逐渐上升。这个布居迁移的能量范围大概是$4k_\text{B}T$的量级(见习题18.1-4,18.1-5,18.1-6).

{\it 能量等于$\mu$的态的占据概率始终是二分之一,$f(\varepsilon,t)$关于$\varepsilon$的函数图像(见图\ref{fig18.1})关于$\varepsilon=\mu,f=\frac{1}{2}$的点反演对称(见习题18.1-6)。}

有了图像上的认识,我们可以讨论预备气体模型中$\mu$对$T$的依赖关系了。明确起见,假定系统有四个 fermion 。此外,假定有两个能级简并$\varepsilon_1=\varepsilon_2$,以及$\varepsilon_3>\varepsilon_2$。在$T=0$时,四个 fermion 都呆在能量$\varepsilon_1(=\varepsilon_2)$的四个轨道态上,而能量$\varepsilon_3$的两个轨道态则是空的。那么 Fermi 能级可以是$\varepsilon_2$和$\varepsilon_3$之间的任意值,当然$\mu$的精确值需要通过取$T\rightarrow 0$的极限来得到。对于充分小的$T$
\begin{equation}
f\equiv \frac{1}{\mathrm e^{\beta(\varepsilon-\mu)}+1}\simeq \begin{cases}
\mathrm e^{-\beta(\varepsilon-\mu)}, & \varepsilon>\mu\& T\simeq 0, \\
1+\mathrm e^{\beta(\varepsilon-\mu)}, & \varepsilon<\mu\& T\simeq 0.
\end{cases}
\end{equation}
从而,在$\varepsilon_1=\varepsilon_2<\varepsilon_3$以及$\tilde N=4$的情况下, \eqref{equ18.6}式对于$T\simeq 0$变为
\begin{equation}
4=4(1-\mathrm e^{\beta(\varepsilon_1-\mu)})+2\mathrm e^{-\beta(\varepsilon_3-\mu)},
\label{equ18.9}
\end{equation}
或者写成
\begin{equation}
\mu = \frac{\varepsilon_1+\varepsilon_3}{2}+\frac{1}{2}k_\text{B}T\ln 2+\dots
\end{equation}
在这个情况下,当$T=0$时$\mu$正好在$\varepsilon_1$和$\varepsilon_3$的中间,并随着$T$线性增大。

我们可以去对比另一个特殊情况,即$\varepsilon_1<\varepsilon_2=\varepsilon_3$。如果依旧有四个fermion,那么$T=0$时Fermi能级($\mu$)应该刚好是$\varepsilon_2$。更有趣的情况是如果只有两个fermion,那么$T=0$时Fermi能级就该出现在$\varepsilon_1$和$\varepsilon_2(=\varepsilon_3)$之间。如同之前所做的,现在\eqref{equ18.9}式对于$T\simeq 0$应该写成
\begin{equation}
2=2(1-\mathrm e^{\beta(\varepsilon_1-\mu)})+4\mathrm e^{-\beta(\varepsilon_3-\mu)},
\end{equation}
以及
\begin{equation}
\mu = \frac{\varepsilon_1+\varepsilon_3}{2}-\frac{1}{2}k_\text{B}T\ln 2+\dots
\end{equation}

不论是哪种情况,Fermi能级总是在远离双简并能级。读者应当能从图\ref{equ18.1}中意识到这个效应,注意$f$在$\varepsilon=\mu$处的中心反演对称性。

从这几种特殊情况出发,我们应当能总结出(对于粒子数$\tilde N$守恒的系统)$\mu$的温度依赖满足这些原则:
\begin{itemize}
\item[(a)] 占据概率仅在$\mu$周围$\Delta\varepsilon\simeq\pm 2k_\text{B}T$附近区域才偏离零或者一。
\item[(b)] 随着$T$升高,Fermi能级$\mu$被高的态密度区域所``排斥''。
\end{itemize}


\noindent{\bf 习题}
\begin{itemize}
\item[18.1-1] 利用对从\eqref{equ18.5}式得到的$\Psi$求导得到预备 fermion 气体模型的平均粒子数。证明这个结果和\eqref{equ18.6}式得到的$\tilde N$一致。
\item[18.1-2] 系统的熵由$S=-k_\text{B}\sum_jf_j\ln f_j$给出,其中$f_j$是系统处在某个微观态上的概率。预备 fermion 气体模型的每个微观态由其六个轨道态是否被占据来决定。
	\begin{itemize}
	\item[a)] 证明这个模型系统总共有$2^{6}=64$个可能的微观态,因而熵的表达式里总共有$64$项。
	\item[b)] 证明这个表达式可以简化成
	\begin{equation*}
	S=-k_\text{B}\sum_{m,n}f_{m,n}\ln f_{m,n},
	\end{equation*}
	这个式子中只有六项。是这个体系的什么性质使得这个神奇的简化可以进行?
	\end{itemize}
\item[18.1-3] 将关于$U$的\eqref{equ17.27}式应用在 fermion 预备气体模型的基本方程之上。证明可以得到与\eqref{equ18.7}式相同的结果。
\item[18.1-4] 证明在$\varepsilon=\mu$处有$\mathrm df/\mathrm d\varepsilon=-\beta/4$。通过这个结果,证明$f$在{\it 大约}$\varepsilon=\mu+k_\text{B}T$处下降至$f=0.25$,在{\it 大约}$\varepsilon=\mu-k_\text{B}T$上升至$f=0.75$(在图\ref{fig18.1}中检查你的结果)。这个大致估算给出了关于$f$迅速变化的$\varepsilon$区域的定性图像。
\item[18.1-5] 证明图\ref{fig18.1}%
\mpar{译注:原文引用图17.2,实际上并没有这张图。}%
(即$f(\varepsilon,T)$作为$\varepsilon$的函数图像)关于点$\varepsilon=\mu,f=\frac{1}{2}$中心反演对称。即,证明$f(\varepsilon,T)$满足如下关系
\begin{equation*}
f(\mu+\Delta,T)=1-f(\mu-\Delta,T),
\end{equation*}
或者是说
\begin{equation*}
f(\varepsilon,T)=1-f(2\mu-\varepsilon,T),
\end{equation*}
并解释为什么这个式子能表达相应的对称性。
\item[18.1-6] 假定$f(\varepsilon,T)$可以用关于$\varepsilon$在如下所述三个区间中的分段线性函数近似。在$\varepsilon\simeq\mu$附近,$f(\varepsilon,T)$近似为通过点$(\varepsilon=\mu,f=\frac{1}{2})$的一条直线,斜率符合该点曲线的情形。对于小的$\varepsilon$,$f(\varepsilon,T)$取做一。对于大的$\varepsilon$,$f(\varepsilon,T)$取做零%
\mpar{译注:原文此处函数自变量写错}%
。\\
在中间段的直线斜率是多少?其``宽度''是多少?将这个结果与习题18.1-4的大致估算相比较。
\end{itemize}

\section{理想Fermi流体}\label{sec18.2}
现在将注意力转向``理想 Fermi 流体'',这个被广泛应用的模型系统具有着深远的意义。理想Fermi气体是经典理想气体的量子对应物,其由无相互作用(或者说充分小以致可略)的 fermion 构成。

概念上讲,最简单的理想 Fermi 流体是一堆中子,在中子星里和重原子核中(作为中子-质子``二分量流体''的一个分量)的情况就是这样。

复合``粒子'',比如说原子,当其包含有奇数个 fermion 时,会体现出 fermion 的性质。那么氦三(\ce{^3He})原子(包含两个质子、一个中子和两个电子)就会像 fermion 一样。相应的,\ce{^3He}原子构成的气体就能处理成``理想 Fermi 流体''。相反,\ce{^4He}原子有一个额外的中子,就表现得像 boson 一样。这两类原子虽然在化学上无法区分,但其在低温下性质的巨大差异,是和这些量子流体的统计力学是分不开的。

金属中的电子是另一类有吸引力的 Fermi 流体,我们将在\ref{sec18.4}节中详细讨论。

我们先来讨论一个一般性的理想 Fermi 流体的统计力学。分析过程将遵循上一节中对待预备 fermion 气体模型的范式。由于流体中轨道态的量子数相当大,并不仅仅是之前的六个,因而求和将被替换成积分。但其他步骤都和之前是完全一样的。

为了计算理想 fermion 流体的基本关系,我们{\it 选择}考虑其与温度$T$的热库和电化学势$\mu$粒子库接触的情形。再强调一遍,实验室中的真实系统可能有着不同的边界条件——它可能是闭的,也可能只是和一个热源热接触,或者其他。但是热力学基本关系并不代表着什么边界条件,我们总能选取使得计算最为方便的边界条件。这里选取了适用于巨正则系综理论的边界条件。

fermion 的轨道态可以用其波函数的波矢$\mathbf k$(回忆\eqref{equ16.43}式)和自旋取向(对于自旋$\frac{1}{2}$的fermion来说是``上''和``下'')。配分求和可以拆分成对于全体轨道态的乘积
\begin{equation}
\mathcal Z=\prod\limits_{{\mathbf k},m_s}z_{{\mathbf k},m_s},
\end{equation}
其中$m_s$可以取两个值,$m_s=\frac{1}{2}$代表着自旋向上,$m_s=-\frac{1}{2}$代表着自旋向下。{\it 每个轨道态都亦或是空的和单占据的。}空轨道态不贡献能量,${{\mathbf k},m_s}$轨道态被占据会贡献能量
\begin{equation}
\varepsilon_{{\mathbf k},m_s}=\frac{p^2}{2m}=\frac{\hbar^2k^2}{2m}\quad (\text{与$m_s$无关}),
\end{equation}
从而对于单个轨道态${{\mathbf k},m_s}$的配分求和是
\begin{equation}
z_{{\mathbf k},m_s}=1+\mathrm e^{-\beta(\hbar^2k^2/2m-\mu)}.
\end{equation}
将$z_{{\mathbf k},1/2}\cdot z_{{\mathbf k},-1/2}$简记为$z_{\mathbf k}$,称为``模式$k$的配分求和''
\begin{equation}
\begin{aligned}
\mathcal Z &= \prod\limits_{{\mathbf k},m_s}z_{{\mathbf k},m_s} = \prod\limits_{\mathbf k}z_{{\mathbf k},1/2}\cdot z_{{\mathbf k},-1/2} \\
&=\prod\limits_k \left[1+2\mathrm e^{-\beta(\hbar^2k^2/2m-\mu)}+\mathrm e^{-2\beta(\hbar^2k^2/2m-\mu)}\right].
\end{aligned}
\end{equation}
这三项分别代表空模式、单占据模式(有两个可能的自旋朝向)、以及双占据模式(一个自旋朝上一个自旋朝下)。

每个轨道态$({{\mathbf k},m_s})$都是独立的,其占据概率为
\begin{equation}
f_{{\mathbf k},m_s}=\frac{\mathrm e^{-\beta(\hbar^2k^2/2m-\mu)}}{z_{{\mathbf k},m_s}}=\frac{1}{\mathrm e^{\beta(\hbar^2k^2/2m-\mu)}+1}.
\label{equ18.17}
\end{equation}
这个函数在图\ref{fig18.1}中已经画出来过了。

到这一步后我们有两条路可以走。原理性算法教导我们要先计算巨正则势$\Psi(=-k_\text{B}T\ln\mathcal Z)$,以得到基本方程。另一条路是直接从\eqref{equ18.17}式出发计算一切感兴趣的量。先来计算基本方程,然后在从``轨道态分布函数''$f_{{\mathbf k},m_s}$出发获取其他(平行的)信息。

巨正则势为
\begin{equation}
\Psi = -k_\text{B}T\sum\limits_{\mathbf k}z_{\mathbf k}=-k_\text{B}T\sum\limits_{\mathbf k}\ln\left[1+\mathrm e^{-\beta(\hbar^2k^2/2m-\mu)}\right]^2.
\label{equ18.18}
\end{equation}

({\it 单}自旋朝向的)轨道态密度是$D(\varepsilon)\,\mathrm d\varepsilon$,在\eqref{equ16.47}式中已经给出
\begin{equation}
D(\varepsilon)\,\mathrm d\varepsilon =\frac{V}{2\pi^2}k^2\frac{\mathrm dk}{\mathrm d\varepsilon}\,\mathrm d\varepsilon = \frac{V}{4\pi^2}\left(\frac{2m}{\hbar^2}\right)^{3/2}\varepsilon^{1/2}\,\mathrm d\varepsilon.
\end{equation}
插入一个两个自旋朝向导致的因子$2$%
\mpar{译注:这个二倍实际上已经体现在\eqref{equ18.18}式对数函数内的平方因子上。}%
,$\Psi$可以写成
\begin{equation}
\begin{aligned}
\Psi &=-2k_\text{B}T\int_0^{\infty}\ln(1+\mathrm e^{-\beta(\varepsilon-\mu)})\,D(\varepsilon)\,\mathrm d\varepsilon \\
&=-k_\text{B}T\frac{V}{2\pi^2}\left(\frac{2m}{\hbar^2}\right)^{3/2}\int_0^{\infty}\varepsilon^{1/2}\ln(1+\mathrm e^{-\beta(\varepsilon-\mu)})\,\,\mathrm d\varepsilon.
\end{aligned}
\label{equ18.20}
\end{equation}
遗憾的是这个积分不能够解析给出。那些由$\Psi$的微分得到的有直接意义的物理量也只能表达成积分形式。这些量可以用数值格点或者其他的近似方法来得到充分高精度的值。原则上讲,对于统计力学而言,问题到\eqref{equ18.20}式已经解决了。

气体中的粒子数$\tilde N$是有意义的。对$\Psi$求导得到
\begin{equation}
\begin{aligned}
\tilde N&= -\frac{\partial\Psi}{\partial\mu}=2\int_0^\infty\frac{1}{\mathrm e^{\beta(\varepsilon-\mu)}+1}D(\varepsilon)\,\mathrm d\varepsilon \\
&=\frac{V}{2\pi^2}\left(\frac{2m}{\hbar^2}\right)^{3/2}\int_0^\infty\frac{\varepsilon^{1/2}}{\mathrm e^{\beta(\varepsilon-\mu)}+1}\,\mathrm d\varepsilon.
\end{aligned}
\label{equ18.21}
\end{equation}
这个式子的第一个形式明显就是对所有态上的占据概率求和。类似的,由求导得到的能量也会完全和对全部态求和$\varepsilon f$的结果相同
\begin{equation}
\begin{aligned}
U&= -\left(\frac{\partial\beta\Psi}{\partial\beta}\right)_{\beta\mu}=2\int_0^\infty\frac{\varepsilon}{\mathrm e^{\beta(\varepsilon-\mu)}+1}D(\varepsilon)\,\mathrm d\varepsilon \\
&=\frac{V}{2\pi^2}\left(\frac{2m}{\hbar^2}\right)^{3/2}\int_0^\infty\frac{\varepsilon^{3/2}}{\mathrm e^{\beta(\varepsilon-\mu)}+1}\,\mathrm d\varepsilon.
\end{aligned}
\label{equ18.22}
\end{equation}

将统计力学应用于量子流体的流程图绘制于表\ref{equ18.1}中。尽管要在稍后的章节中才会仔细考虑, Bose 流体也包含在内。它们之间的区别仅仅是几个符号,正如在\ref{sec18.5}节中会出现的一样。

\begin{table}
\caption{{\bf 量子流体的统计力学}。上面的符号对应fermion,下面的符号对应boson。 \\ 
(a) 配分求和因子。自旋取向数为$g_0=2S+1$(对于零自旋boson $g_0=1$;对于自旋$\frac{1}{2}$ fermion $g_0=2$,等等)。\\
(b) $z_{\mathbf k}$是(给定$\mathbf k$和$m_s$的)单轨道态的配分求和。\\
(c) $f_{{\mathbf k},m_s}$是轨道态${{\mathbf k},m_s}$的平均占据数(或``占据概率'')。\\
(d, e,\& f) $D(\varepsilon)$是单个自旋朝向轨道态密度。\\
(g) $\Psi(T,\mu)$是基本关系。\\
(h, i \& j)$P=P(U,V)$是一个状态方程,对于fermion和boson都适用。}
\begin{tabular}{cc}
\toprule
$\mathcal Z=\prod_{{\mathbf k},m_s}z_{{\mathbf k},m_s}=\prod_{{\mathbf k}}z^{g_0}_{{\mathbf k}}$ & (a) \\
$z_{\mathbf k}=\left[1\pm\mathrm e^{-\beta(\varepsilon_{\mathbf k}-\mu)}\right]^{\pm 1}$ & (b) \\
$f_{{\mathbf k},m_s}=\frac{1}{\mathrm e^{\beta(\varepsilon_{\mathbf k}-\mu)}\pm 1}$ & (c) \\
$-\beta\Psi=\ln\mathcal Z=g_0\sum_{\mathbf k}\ln z_{\mathbf k}=\pm g_0\sum_{\mathbf k}\ln[1\pm\mathrm e^{-\beta(\varepsilon_k-\mu)}]$ & (d) \\
$=\pm g_0\int_0^\infty\ln[1\pm\mathrm e^{-\beta(\varepsilon-\mu)}]D(\varepsilon)\,\mathrm d\varepsilon$ & (e) \\
$D(\varepsilon)=\frac{V}{(2\pi)^2}\left(\frac{2m}{\hbar^2}\right)^{3/2}\varepsilon^{1/2}$ & (f) \\
$\Psi=-\frac{2}{3}\frac{g_0V}{(2\pi)^2}(\frac{2m}{\hbar^2})^{3/2}\int_0^\infty\frac{\varepsilon^{3/2}}{\mathrm e^{\beta(\varepsilon-\mu)}\pm 1}\,\mathrm d\varepsilon\quad$(基本方程)& (g) \\
注意到 $\Psi=-\frac{2}{3}\int_0^\infty\varepsilon f(\varepsilon)g_0D(\varepsilon)\,\mathrm d\varepsilon=-\frac{2}{3}U$ & (h) \\
以及 $\Psi=-PV\quad$ (对于简单系统)& (i) \\
故 $U=\frac{2}{3}PV\quad$ (状态方程)&(j) \\
\bottomrule
\end{tabular}
\label{tab18.1}
\end{table}

在仔细讨论这些一般性结果之前,先确认其在高温下能退化成经典理想气体是明智的,顺便也能确定区分经典和量子力学的分界线。

\noindent{\bf 习题}
\begin{itemize}
\item[18.2-1] 对于 fermion 证明表18.1中的c, g, h, i以及j式。
\end{itemize}

\section{经典极限和量子修正}\label{sec18.3}
量子区域的特征在于一个 fermion 不能去占据任意选定的轨道态,而要去看这个轨道态上是否事先被其他粒子所占据。但是在高温或者低密度下,不管哪个轨道态被占据的概率都是很小的,那么 fermion 无法同时占据同一个轨道态所造成的影响会很小。在低密度或高温的情形下,少数粒子分布在大量的态上,使得所有气体都变得经典起来。

能量为$\varepsilon$的态被占据的概率是$[\mathrm e^{\beta(\varepsilon-\mu)}+1]^{-1}$,当$\mathrm e^{-\beta\mu}$很大的时候,或者说{\it 逸度(fugacity)}$\mathrm e^{\beta\mu}$很小的时候,这个概率(对于全部$\varepsilon$)会很小:
\begin{equation}
\mathrm e^{\beta\mu} \ll 1 \quad\text{(经典区域)}.
\end{equation}
在经典区域中,相应的概率约化为
\begin{equation}
f_{{\mathbf k},m_s}z_{{\mathbf k},m_s}\simeq\mathrm e^{\beta\mu}\mathrm e^{-\beta\varepsilon}.
\label{equ18.24}
\end{equation}
在图\ref{fig18.1}中,,经典区域对应于Fermi 能级衰退为相当大的负值,从而所有物理轨道都呆在$f(\varepsilon,T)$曲线的``尾巴''上。

我们首先确认\eqref{equ18.24}式能够重新得到经典的结果,然后讨论什么样的物理条件会导致一个小的逸度。

\eqref{equ18.21}式给出的粒子数$\tilde N$对于小的逸度变成
\begin{equation}
\tilde N\simeq\frac{g_0V}{(2\pi)^2}\left(\frac{2m}{\hbar^2}\right)^{3/2}\mathrm e^{\beta\mu}\int_0^\infty\mathrm e^{-\beta\varepsilon}\varepsilon^{1/2}\,\mathrm d\varepsilon=\frac{g_0V}{\lambda_T^3}\mathrm e^{\beta\mu},
\label{equ18.25}
\end{equation}
式中$\lambda_T$(马上就会解释它的物理意义)由下式定义
\begin{equation}
\lambda_T=\frac{h}{\sqrt{2\pi mk_\text{B}T}},
\label{equ18.26}
\end{equation}
以及$g_0=2S+1$是可能的自旋取向数目(对于自旋$\frac{1}{2}$来说是二)。对于由\eqref{equ18.22}式给出的能量来说也是类似的
\begin{equation}
U=\frac{2V}{(2\pi)^2}\left(\frac{2m}{\hbar^2}\right)^{3/2}\mathrm e^{\beta\mu}\int_0^\infty\mathrm e^{-\beta\varepsilon}\varepsilon^{3/2}\,\mathrm d\varepsilon=\frac{3}{2}k_\text{B}T\frac{g_0V}{\lambda_T^3}\mathrm e^{\beta\mu},
\label{equ18.27}
\end{equation}
从而得到
\begin{equation}
U=\frac{3}{2}\tilde Nk_\text{B}T
\end{equation}
这就是熟知的理想气体的状态方程。此外\eqref{equ18.25}式和\eqref{equ18.27}式也分别对经典理想气体适用。

在确保了 Fermi 气体在经典极限下能正常表现后,我们开始探讨划分经典和量子区域的判据。从讨论中可以知道这个分界线出现在逸度大致是一的量级时
\begin{equation}
\mathrm e^{\beta\mu}\simeq 1\quad\text{(经典-量子边界)},
\end{equation}
或者从\eqref{equ18.25}式来看
\begin{equation}
\lambda^3_T/\left(\frac{g_0V}{\tilde N}\right)\simeq 1\quad\text{(经典-量子边界)}.
\end{equation}

``量子判据''对$\lambda_T$的意义给出了一个图像上的解释。事实上$\lambda_T$就是一个具有动能$k_\text{B}T$的粒子的量子波长(见习题18.3-2),其被称为``热波长''。从\eqref{equ18.25}式可以知道{\it 在经典极限下逸度等于``热体积''$\lambda_T^3$与单个(有给定自旋的)粒子占据体积$V/(\tilde N/g_0)$之比。若热体积大于(有给定自旋的)粒子平均占据的体积,这个体系即处于量子区域,这可能是因为大的$\tilde N$或者低的$T$(从而导致大的$\lambda_T$)。}


\noindent{\bf 习题}
\begin{itemize}
\item[18.3-1] 通过变量替换$\varepsilon=x^2$计算由\eqref{equ18.25}和\eqref{equ18.27}%
\mpar{译注:原文为18.26式,有误}%
式所给出的积分,注意可以通过先对$\beta$求导来得到一个较简单的积分式。
\item[18.3-2] 利用量子力学的公式$p=h/\lambda$求得与波长相应的动量,并比较能量$p^2/2m$和$k_\text{B}T$,来验证将$\lambda_T$诠释成``热波长''的合理性。
\end{itemize}

\section{强量子区:金属中的电子}\label{sec18.4}
第一眼看起来,金属中的电子不是一个理想 Fermi 流体的好例子,电子携带的电荷表观上会带来相当强的粒子间相互作用力。然而离子实的正电荷背景至少在平均意义上中和了电子的负电荷。库伦势的长程性让平均效应占据了主导地位,某一点的电势是是大量的电子和正离子效果的总和——有些就在附近,但更多的来自于遥远的地方。所有这些都能够被量化,其近似的精确性可以通过固体物理学的方法论去控制。我们暂且先简单地接受金属中电子作为理想费米气体这个模型,而不论其基础是否牢固%
\mpar{译注:在学习了量子统计物理,对元激发、准粒子等概念有所了解之后,读者会对这个问题有新的认识。}%
。

对Fermi能级的估算(马上就会去做)会告诉我们对于那些合理的温度都有$\mu\gg k_\text{B}T$。从而金属中的电子将是强量子区域中的理想费米气体的一个例子。这一节通过金属中的电子来分析强量子区域中的理想 Fermi 气体,只是为了找一个物理背景,而实际上所讨论的显然不仅限于此。

首先考虑零温下的电子态,将$T=0$下的Fermi能级记做$\mu_0$(``费米能量'')。在$\varepsilon<\mu_0$上的占据概率是一,而$\varepsilon>\mu_0$上的占据概率是零,从而(据\eqref{equ18.21}式)
\begin{equation}
\tilde N=\frac{\sqrt{2}m^{3/2}V}{\pi^2\hbar^3}\int_0^{\mu_0}\varepsilon^{1/2}\,\mathrm d\varepsilon =\frac{(2m)^{3/2}V}{3\pi^2\hbar^3}\mu_0^{3/2},
\end{equation}
即
\begin{equation}
\mu_0=\frac{\hbar^2}{2m}\left(3\pi^2\frac{\tilde N}{V}\right)^{2/3}.
\label{equ18.32}
\end{equation}
金属导体中单位体积内的电子数大概是\SI{e22}{\per\cubic\centi\meter}到\SI{e22}{\per\cubic\centi\meter}的量级(对应于每个离子贡献一到两个电子,离子间距$\simeq\SI{5}{\angstrom}$)。从而金属中电子的Fermi能量$\mu_0$(或者``Fermi 温度''$\mu_0/k_\text{B}$)大概是
\begin{equation}
\frac{\mu_0}{k_\text{B}}\simeq\SI{e4}{\kelvin}\sim\SI{e5}{\kelvin}.
\end{equation}
前面所提到的其他一些Fermi流体其Fermi温度还会更高——白矮星里的电子是\SI{e9}{\kelvin},而中子星以及重核内的中子是\SI{e12}{\kelvin}。

这么高的Fermi温度意味着极高的电子能量。零温下的能量是
\begin{equation}
U(T=0)=2\int_0^{\mu_0}\varepsilon D(\varepsilon)\,\mathrm d\varepsilon=\frac{3}{5}\tilde N\mu_0.
\label{equ18.34}
\end{equation}
从而粒子的平均能量是$\frac{3}{5}\mu_0$,相应于\SI{e4}{\kelvin}的温度。

随着温度上升,Fermi能级下降(被高能量区域的高密度态``排斥'',在\ref{sec18.1}节中讨论过了这个现象)。进而一些电子从能量低于$\mu$的位置``升级''到了能量高于$\mu$的位置,从而抬高了系统的能量。为了定量研究这些效应,我们需要借用关于积分$\int \phi(\varepsilon)f(\varepsilon,T)\,\mathrm d\varepsilon$的一个一般性结果,其中$\phi(\varepsilon)$是一个任意函数,$f(\varepsilon,T)$是Fermi占据概率。利用$f(\varepsilon,T)$在低温下近似为阶跃函数的事实,可以将这个积分展开成温度的幂级数
\begin{equation}
\begin{aligned}
\int_0^\infty\phi(\varepsilon)f(\varepsilon,T),\mathrm d\varepsilon=&\int_0^\mu\phi(\varepsilon)\,\mathrm d\varepsilon+\frac{\pi^2}{6}(k_\text{B}T)^2\phi'(\mu)\\
&+\frac{7\pi^4}{360}(k_\text{B}T)^4\phi'''(\mu)+\dots
\end{aligned}
\label{equ18.35}
\end{equation}
其中$\phi'$和$\phi'''$分别是$\phi$关于$\varepsilon$在$\varepsilon=\mu$处的一阶和三阶导数。应当注意这里$\mu$是温度依赖的(而不是零温Fermi能量$\mu_0$)。

首先处理Fermi能量的温度依赖。其可由\eqref{equ18.21}给出%
\mpar{译注:原文下式有误。}
\begin{equation}
\tilde N=2\int_0^\infty f(\varepsilon,T)D(\varepsilon)\,\mathrm d\varepsilon=\frac{V}{2\pi^2}\left(\frac{2m}{\hbar^2}\right)^{3/2}\int_0^\infty\varepsilon^{1/2}\,\mathrm d\varepsilon .
\end{equation}
在\ref{equ18.35}式中取$\phi(\varepsilon)=\varepsilon^{1/2}$
\begin{equation}
\tilde N=\frac{V}{3\pi^2}\left(\frac{2m}{\hbar^2}\right)^{3/2}\mu^{3/2}\left[1+\frac{\pi^2}{8}\left(\frac{k_\text{B}T}{\mu}\right)^2+\dots\right].
\label{equ18.37}
\end{equation}
在零温下回到了\eqref{equ18.32}式。为了得到关于温度的二阶项,在二阶项里把$\mu$替换成$\mu_0$就足够了,从而
\begin{equation}
\mu(T)=\mu_0\left[1-\frac{\pi^2}{12}\left(\frac{k_\text{B}T}{\mu_0}\right)+\dots\right].
\label{equ18.38}
\end{equation}
这个结果和Fermi能级随温度上升而下降的预期相符。但对于典型的$\mu_0/k_\text{B}$的值(\SI{e4}{\kelvin}的量级),室温下Fermi能级相对于零温仅降低大概\SI{0.1}{\percent}的水平!

能量的处理是类似的,仅需要将$\varepsilon^{1/2}$替换成$\varepsilon^{3/2}$,结果是
\begin{equation}
U=\frac{V}{5\pi^2}\left(\frac{2m}{\hbar^2}\right)^{3/2}\mu^{5/2}\left[1+\frac{5}{8}\pi^2\left(\frac{k_\text{B}T}{\mu}\right)^2+\dots\right].
\label{equ18.39}
\end{equation}
相比于\eqref{equ18.32}式,可以验证在$T=0$下我们重新得到了$U=\frac{3}{5}\tilde N\mu_0$(\eqref{equ18.34}式)。这让我们忍不住用\eqref{equ18.39}式去除\eqref{equ18.37}式,得到
\begin{equation}
U=\frac{3}{5}\tilde N\mu\left[1+\frac{1}{2}\pi^2\left(\frac{k_\text{B}T}{\mu}\right)^2+\dots\right].
\end{equation}
将$\mu(T)$用\eqref{equ18.38}式替换,最后得到
\begin{equation}
U=\frac{3}{5}\tilde N\mu_0\left[1+\frac{5}{12}\pi^2\left(\frac{k_\text{B}T}{\mu_0}\right)^2+\dots\right],
\end{equation}
热容为
\begin{equation}
C=\frac{3}{2}\tilde Nk_\text{B}\left(\frac{\pi^2}{3}\frac{k_\text{B}T}{\mu_0}\right)+O(T^3).
\label{equ18.42}
\end{equation}
前因子$\frac{3}{2}\tilde Nk_\text{B}$是经典结果,而括号里的因子就是源自于 fermion 量子性质的``量子修正因子''。室温下量子修正因子大致是$\frac{1}{10}$的量级(对于$\frac{\mu_0}{k_\text{B}}\simeq\SI{e4}{\kelvin}$)。热容相对于经典结果的这个锐减和几乎所有金属的实验结果都精确相符。

为了将金属热容的观测结果与理论相对照,我们得别忘了(\ref{sec16.6}节中)晶格振动对热容贡献一项正比于$T^3$的值,加上电子所贡献的线性项和三次项
\begin{equation}
C=AT+BT^3+\dots
\end{equation}
系数$A$由\eqref{equ18.42}式给出,而$B$等于\eqref{equ18.42}式中的三次项加上 Debye 理论给出的系数(主导项)。将实验数据以$C/T$与$T^2$的形式绘图%
\mpar{从而能看到数据都在一条直线上。}%
,那么系数$A$就是$T=0$处的截距,系数$B$是相应直线的斜率。事实上用实验数据这样作图真能得到相当完美的直线,而系数$A$和$B$也与\eqref{equ18.42}式和 Debye 理论\eqref{equ16.51}式完美的吻合%
\mpar{译注:读者应当对热力学实验的精确度有恰当的认识,不要因为作者的提法而对真实的结果有过高的期望。}%
。

\eqref{equ18.42}式给出的热容可以半定量的理解。当温度从$T=0$往上升高时,电子从能量稍稍比$\mu_0$低一点地方``升级''到了比$\mu_0$稍稍高一点的地方。这个布居数的迁移主要发生在$2k_\text{B}T$那么大的能量范围内(回忆图\ref{fig18.1}和习题18.1-7)。升级过去的电子数大概是$D(\mu_0)2k_\text{B}T$的量级,每个电子大概提升了$k_\text{B}T$的能量。从而能量的增量大致是
\begin{equation}
U-U_0\simeq 2D(\mu_0)(k_\text{B}T)^2.
\end{equation}
又有$D(\mu_0)=3\tilde N/2\mu_0$,故
\begin{equation}
U-U_0\simeq\frac{3\tilde N(k_\text{B}T)^2}{\mu_0},
\end{equation}
以及
\begin{equation}
C\simeq \frac{3}{2}\tilde Nk_\text{B}\left(2\frac{k_\text{B}T}{\mu_0}\right).
\label{equ18.46}
\end{equation}
这个粗糙的估算和\eqref{equ18.42}式给出的定量计算结果已经很接近了,相差的只是括号里的因子,之前是$\pi^2/3$而\eqref{equ18.46}式中是$2$。

\noindent{\bf 习题}
\begin{itemize}
\item[18.4-1] 证明\eqref{equ18.32}式可以解释成$\mu_0=\hbar^2k_F^2/2m$,其中$k_F$是一个八分之一能包含$2\tilde N$粒子(回忆\ref{sec16.6}节的结果)的$\bf$-空间中的球的半径。为什么这里是$2\tilde N$而不是$\tilde N$?
\item[18.4-2] 通过如下方式导出\eqref{equ18.35}式%
\mpar{译注:另一种更形式化的证明,可以参考 Blacnkenbecler, R. {\it Am. J. Phys.} {\bf 25}, 279}%
:
	\begin{itemize}
	\item[a)] 将\eqref{equ18.35}式的积分记做$I$,首先分部积分一次,并令$\Phi\equiv\int_0^\varepsilon \phi(\varepsilon')\,\mathrm d\varepsilon'$。再将$\Phi(\varepsilon)$展开成$(\varepsilon-\mu)$的幂级数到三阶,证明
	\begin{equation*}
	I=-\sum\limits_{m=0}^\infty \frac{1}{m!}\frac{\mathrm d^m\Phi(\mu)}{\mathrm d\mu^m}I_m,
	\end{equation*}
	其中
	\begin{equation*}
	I_m=\int_0^\infty(\varepsilon-\mu)^m\frac{\mathrm df}{\mathrm d\varepsilon}\,\mathrm d\varepsilon = -\beta^{-m}\int_{-\beta\mu}^\infty\frac{\mathrm e^x}{(\mathrm e^x+1)^2}x^m\,\mathrm dx.
	\end{equation*}
	\item[b)] 证明将积分下限替换成$-\infty$只会产生一个指数小的误差,从而所有$m$的奇次项都为零。
	\item[c)] 计算头两阶非零项,证明其与\eqref{equ18.35}式相符。
	\end{itemize}
\end{itemize}

\section{理想 Bose 流体}\label{sec18.5}
理想 Bose 流体的公式和理想 Fermi 流体的公式长得几乎一模一样。不论是表\ref{tab18.1}中所期待的还是等会将会检验的,两个公式仅仅只差一个符号。但是其所导致的结果却大为不同。低温下fermion倾向于填满某个Fermi能量下所有的轨道态,而boson则倾向于凝聚到一个最低能量的轨道态上。这个凝聚过程在一个``凝聚温度''附近狭窄的区域中突然发生。相变会导致\ce{^4He}的超流性(对于fermion流体\ce{^3He}不存在这个现象),以及铅和许多金属中的超导性。

考察具有整数自旋的复合粒子构成的理想费米流体。自旋朝向的数目是$g_0=2S+1$,其中$S$是自旋的大小。

和fermion的情形一样,boson可能的轨道态由$\mathbf k$和$m_s$标记,其巨正则配分求和也可以对轨道态作分解(表\ref{tab18.1}中的(a)行)。

对于单个轨道态的配分求和与$m_s$无关,即,对于{\it 每个}$m_s$
\begin{equation}
\begin{aligned}
z_{\mathbf k}&=z_{{\mathbf k},m_s}=1+\mathrm e^{-\beta(\varepsilon_k-\mu)}+\mathrm e^{-\beta(2\varepsilon_k-2\mu)}+\mathrm e^{-\beta(3\varepsilon_k-3\mu)}+\dots
&=\frac{1}{1-\mathrm e^{-\beta(\varepsilon_k-\mu)}}.
\end{aligned}
\label{equ18.47}
\end{equation}
这验证了表\ref{tab18.1}中的(b)行。

轨道态${\mathbf k},m_s$上的平均boson数目为
\begin{equation}
\begin{aligned}
\bar{n}_{{\mathbf k},m_s}&=\left[\mathrm e^{-\beta(\varepsilon_k-\mu)}+2\mathrm e^{-\beta(2\varepsilon_k-2\mu)}+3\mathrm e^{-\beta(3\varepsilon_k-3\mu)}+\dots\right]/z_{{\mathbf k},m_s} \\
&=k_\text{B}T\frac{\partial}{\partial\mu}\ln z_{\mathbf k,m_s},
\end{aligned}
\end{equation}
这个式子很像$\beta N=\partial/\partial\mu\ln\mathcal Z$,但它现在是对单个轨道态适用的。计算这个导数,有
\begin{equation}
\bar{n}_{{\mathbf k},m_s}\equiv f_{{{\mathbf k},m_s}}=\frac{1}{\mathrm e^{\beta(\varepsilon_k-\mu)}-1},
\end{equation}
这个结果在表\ref{tab18.1}的(c)行中列出。请注意,相比于fermion的情况,这里$f_{{\mathbf k},m_s}$不一定得小于(或等于)一。$f_{{\mathbf k},m_s}$这个量常被称为``占据概率'',但更合适的称呼是``平均占据数''$\bar{n}_{{\mathbf k},m_s}$。

稍稍想想$\bar{n}_{{\mathbf k},m_s}$的形式就能知道Bose粒子构成的气体的摩尔Gibbs函数一定得是负的。这是因为如果$\mu$是正的话,那么能量$\varepsilon_{\mathbf k}$等于$\mu$的那个轨道态上的占据数将是无穷大!由此我们有结论,对于粒子数有界的气体(并且通过选取能量零点使基态能量为零),{\it 摩尔Gibbs势$\mu$总得是负的}。

$\bar n$关于$\beta(\varepsilon-\mu)$的函数曲线绘于图\ref{fig18.2}中。占据数从$\varepsilon=\mu$处的无穷大降至$\varepsilon=\mu+0.693k_\text{B}T$处的一。在子图中,大致比对了$\mu$作为$\varepsilon$的函数在不同温度下($T_2<T_1$)以及不同$\mu$下的两条曲线。

\begin{figure}[htbp]
\centering
\includegraphics[width=\textwidth]{Pictures/fig18.2.png}
\figcaption{在给定的$T$和$\mu$下Bose平均占据数$\bar n$随轨道态能量$\varepsilon$的变化。子图大致比对了$T_2<T_1$以及$\mu_2<\mu_1$的两条曲线。}
\label{fig18.2}
\end{figure}

如果所考察的系统与粒子库相接触,即$\mu$为常数,那么子图中的$\bar{n}(\varepsilon,T_2)$就会向右移动。这个系统中的粒子数随着温度升高。如果所考察的系统保持着不变的粒子数,即$\bar{n}(\varepsilon,T)$的积分为常数。如图所示,摩尔Gibbs势$\mu$就会随着温度升高而降低(和Fermi气体中的情形一样)。

巨正则势$\Psi$对$\mathcal Z$的对数,而它又等于\eqref{equ18.47}式中给出的$z_{{\mathbf k},m_s}$相连乘。从而,如表\ref{tab18.1}中的(d)至(g)行
\begin{equation}
\beta\Psi=g_0\int_0^\infty \ln[1-\mathrm e^{-\beta(\varepsilon-\mu)}]D(\varepsilon)\,\mathrm d\varepsilon,
\end{equation}
或者说,通过分部积分
\begin{equation}
\Psi=-\frac{2}{3}\frac{g_0V}{(2\pi)^2}\left(\frac{2m}{\hbar^2}\right)^{3/2}\int_0^\infty\frac{\varepsilon^{3/2}}{\mathrm e^{\beta(\varepsilon-\mu)}-1}\,\mathrm d\varepsilon,
\label{equ18.51}
\end{equation}
便又一次得到了体系的力学状态方程$P=2U/3V$(表\ref{tab18.1}中的(i)和(j)行)%
\mpar{译注:利用 Ehrenfest 定理和 Virial 定理可以更形式化的证明(j)式。}%
。

对于一个与粒子库接触从而$\mu$为常数的系统,其热力学可以直接的得到。但对于粒子数$\tilde N$为常数的系统,上面列出的那些看似无害的方程会导致一些惊人的结论,这些结论无法在 fermion 系统或者经典系统中找到。作为这些思考的预先练习,我们先来看看一个物理上粒子数就不守恒的系统。

\section{粒子数不守恒的理想Bose流体:重探电磁辐射}\label{sec18.6}
正如在\ref{sec18.1}中所认识到的,boson对应于经典物理中的``波''。这个经典对应使其不必像fermion一样,boson的粒子数不一定得是守恒的%
\mpar{译注:然而,fermion的粒子数也不一定是守恒的,正负电子湮灭便是一个例子。实际上,这取决于能标以及相互作用的形式。}%
在某些情况下,例如\ce{^4He}原子构成的流体,boson是守恒的;在另一些情况下,如``光子气体''(回忆\ref{sec3.6}节),boson不是守恒的。存在着类似于两个光子通过某种非线性的耦合产生三个光子的过程%
\mpar{译注:例举的这个过程需要其他物质的参与。}%
。我们怎样才能将理想Bose气体的公式应用到这可能会出现不守恒的情形中呢?

回到\ref{sec17.2}和\ref{equ17.3}节中对于巨正则系综理论的讨论。在那儿我们在能量(\eqref{equ17.30}式)和粒子数(\eqref{equ17.31}式式)的约束下,对无序性取极大。为处理这两个约束,引入了Lagrange参数$\lambda_2,\lambda_3$(\eqref{17.33}式),随后通过物理上的考虑确认其分别为$\lambda_2=\beta$以及$\lambda_3=\beta\mu$。{\it 处理不守恒的粒子只要简单的忽略对粒子数的约束关系就行了。}略去$\lambda_3$相当于取$\lambda_3=0$或者说$\mu=0$。由此我们有结论,{\it 粒子数不守恒的Bose气体的摩尔Gibbs势为零}%
\mpar{译注:严格来说,讨论这类物质的摩尔Gibbs势没有意义。}%
。

$\mu=0$的情况下巨正则系综理论和正则系综理论一模一样。从而对光子气体的巨正则分析便是\ref{sec16.7}节中所发展的对电磁辐射正则处理的简单重复。读者应当将表\ref{tab18.1}和\ref{sec16.7}一步一步的平行比较(习题18.6-2)。

考察\ref{sec18.6}节中和本节中所用的不同观点将是有教益的。在之前的分析中我们关注于{\it 电磁场的简正模式},从而在正则系综理论下处理。本节中我们关注于{\it 场量子},或者说{\it 光子},使用巨正则理论是更为自然的。但粒子数不守恒导致了$\mu$为零,进而使得两套理论完全等价。只有用的语言不一样!

能量为$\varepsilon$的光子数为$(\mathrm e^{\beta\varepsilon}-1)^{-1}$,其容许的能量为
\begin{equation}
\varepsilon=\hbar\omega=\hbar c\frac{2\pi}{\lambda}=\frac{hc}{\lambda}.
\end{equation}
这里$c$是光速,$\lambda$是光子的量子力学波长(或者在\ref{sec16.7}节的模式语境下,简正模式的波长)。无穷长波长的boson的布居数是无界的%
\footnote{自然,具有无穷长波长的光子只能呆在无穷大的容器中,但是充分大的容器中的光子数可以大于任何预先设定的上界。}%
这些长波光子的能量为零,故随着boson粒子数在形式上发散,能量不会发散。

重申一遍,电磁辐射即可以用简正模式来描述,也可以用这些模式的量子激发来描述。前一个观点导向正则系综。后一个观点导向粒子数不守恒的Bose气体、这类气体的摩尔Gibbs势为零,以及基态上布居着的无穷多个(不可观测的)零能boson。

所有的这些看起来可能会相当的不自然和怪异,它们在粒子数{\it 守恒}的boson系统中没有对应物。而接下来我们将转向这类系统,其会展现出\ce{^4He}超流及金属超导等新物理。

\noindent{\bf 习题}
\begin{itemize}
\item[18.6-1] 计算温度为\SI{300}{\kelvin}下在\SI{1}{\cubic\meter}大小的方盒子里能量最低的轨道态上的光子数。其总能量为多少?在波长为\SI{5000}{\angstrom}的单个轨道态上的光子数是多少,这些光子的总能量呢?
\item[18.6-2] 
	\begin{itemize}
	\item[(a)] 将巨正则系综理论应用至光子气体时,能否在表\ref{tab18.1}中(f)式使用轨道态密度函数$D(\varepsilon)$?解释之。
	\item[(b)] 记光速为$c$,证明由$c=\text{波长}/\text{周期}$可以得到$\omega=ck$。从这个关系式以及\ref{sec16.5}节的内容出发求得轨道态密度$D(\varepsilon)$。
	\item[(c)] 证明用巨正则系综理论对光子气体的分析和\ref{sec16.7}节中给出的理论完全一致。
	\end{itemize}
\end{itemize}
\section{Bose凝聚}\label{sec18.7}
小插曲结束了,现在来面对正题:粒子数守恒的封闭系统。正如图\ref{fig18.2}及其后续的讨论中所展现的,摩尔Gibbs势$\mu$会随着温度下降而上升(同fermion的情形一致)。

假定作为物质粒子的boson动能写成$\varepsilon=p^2/2m$,其轨道态密度正比于$\varepsilon^{1/2}$(表\ref{tab18.1}中的(f)式),粒子数为
\begin{equation}
\tilde N_e = \frac{g_0V}{(2\pi)^2}\left(\frac{2m}{\hbar^2}\right)^{3/2}\int_0^\infty \frac{\varepsilon^{1/2}}{\xi^{-1}\mathrm e^{\beta\varepsilon}-1}\,\mathrm d\varepsilon,
\label{equ18.53}
\end{equation}
其中$\xi$是逸度(fugacity)
\begin{equation}
\xi \equiv\mathrm e^{\beta\mu},
\end{equation}
$\tilde N_e$的下标$e$的意义随后会解释,现在把其当成$\tilde N$的另一个记号就好。摩尔Gibbs势总是负的(对于粒子数守恒的系统),从而逸度是零到一之间的一个量。
\begin{equation}
0<\xi<1.
\end{equation}
这个事实鼓励我们将\eqref{equ18.53}式按逸度的幂次展开,得到
\begin{equation}
\tilde N_e=\left[\frac{g_0V}{(2\pi)^2}\left(\frac{2m}{\hbar^2}\right)^{3/2}\right]\frac{\sqrt{\pi}}{2}(k_\text{B}T)^{3/2}F_{3/2}(\xi)=\frac{g_0V}{\lambda_T^3}F_{3/2}(\xi),
\label{equ18.56}
\end{equation}
其中$\lambda_T$为``热波长''(见\eqref{equ18.26}式),以及
\begin{equation}
F_{3/2}(\xi)=\sum_{r=1}^\infty\frac{\xi^r}{r^{3/2}}=\xi+\frac{\xi^2}{2\sqrt(2)}+\frac{\xi^3}{3\sqrt{3}}+\dots
\label{equ18.57}
\end{equation}
在高温极限下逸度很小,从而$F_{3/2}(\xi)$可以用$\xi$(它的领头项)来代替,从而\eqref{equ18.56}式退化至用其经典形式\eqref{equ18.25}式。

类似的,
\begin{equation}
U=\left[\frac{g_0V}{(2\pi)^2}\left(\frac{2m}{\hbar^2}\right)^{3/2}\right]\frac{3\sqrt{\pi}}{4}(k_\text{B}T)^{5/2}F_{5/2}(\xi)=\frac{3}{2}k_\text{B}T\frac{g_0V}{\lambda_T^3}F_{3/2}(\xi),
\label{equ18.58}
\end{equation}
其中
\begin{equation}
F_{5/2}(\xi)=\sum_{r=1}^\infty\frac{\xi^r}{r^{5/2}}=\xi+\frac{\xi^2}{4\sqrt(2)}+\frac{\xi^3}{9\sqrt{3}}+\dots
\end{equation}
如果将$F_{5/2}(\xi)$替换成$\xi$,$U$的式子也能退化到经典极限\eqref{equ18.27}式。

用\eqref{equ18.56}式去除\eqref{equ18.58}式
\begin{equation}
U=\frac{3}{2}\tilde N_ek_\text{B}T\frac{F_{5/2}(\xi)}{F_{3/2}(\xi)},
\label{equ18.60}
\end{equation}
从而比值$F_{5/2}(\xi)/F_{3/2}(\xi)$便衡量了相对于经典状态方程的差异。

$F_{5/2}(\xi)$和$F_{3/2}(\xi)$这两个函数项级数的所有系数都是正的,从而它们都是$\xi$的单调递增函数,如图\ref{fig18.3}所示。其在$\xi=0$处的斜率都是一。在$\xi=1$处函数$F_{3/2}(\xi)$和$F_{5/2}(\xi)$的值分别是$2.612$和$1.34$。

\begin{figure}
\includegraphics[width=\textwidth]{Pictures/fig18.3.png}
\figcaption{函数$F_{5/2}(\xi)$和$F_{3/2}(\xi)$分别用以表征粒子数守恒的boson气体的粒子数和能量(\eqref{equ18.57}-\eqref{equ18.60}式)。}
\label{fig18.3}
\end{figure}

这两个函数满足关系式
\begin{equation}
\frac{\mathrm d}{\mathrm d\xi}F_{5/2}(\xi)=\frac{1}{\xi}F_{3/2}(\xi),
\end{equation}
由这个我们知道$F_{5/2}(\xi)$在$\xi=1$处的斜率等于$F_{3/2}(1)$,或者说$2.612$。而$F_{3/2}(\xi)$在$\xi=1$处的斜率是无穷大(习题18.7-2)。

对于给定气体的分析现在明确了。假定$\tilde N_e,V$和$T$是已知的。那么可得$F_{3/2}(\xi)=N_e\lambda_T^3/g_0V$,逸度$\xi$可从图\ref{fig18.3}中得到。给定逸度之后所有热力学函数都能由巨正则系综理论给出。例如,能量就可以通过图\ref{fig18.3}以及\eqref{equ18.58}或\eqref{equ18.60}式中求得。

上面这些讨论看上去都非常直接并且有道理,但如果一旦考虑到对于某些给定的$\tilde N_e,V$的$T$的值,会使得$\tilde N_2\lambda_T^3/g_0V$这个量大于$2.612$时,可能就会有点懵逼了。从图\ref{fig18.3}中找不到逸度$\xi$的解!上述分析在``极端量子极限''下失效了!

停下来仔细找找问题出在了哪儿。当$\tilde N_e\lambda_T^3/g_0V(=F_{3/2}(\xi))$等于$2.612$时,逸度会变成一,或者说摩尔Gibbs势变成零。但在之前我们就注意到了当$\mu=0$时零能轨道态上的布居数$\bar n$会发散。当将对轨道态的求和转换成积分时(用轨道态密度加权,其在$\mu=0$处为零),这个病态的行为会消失。这种操作在$g_0V/N_e\lambda_T^3<2.612$时是可接受的,但如果这个量大于$2.612$的话,我们得在用积分替代求和时进行精细和微妙的处理。

先别管该怎么处理$g_0V/\tilde N_e\lambda_T^3\ge 2.612$的情况,先算算``积分分析''(相对于``求和分析'')失效的温度。取$g_0V/\tilde N_e\lambda_T^3=2.612$有
\begin{equation}
k_\text{B}T_c=\frac{2\pi\hbar^2}{m}\left(\frac{1}{2.612}\frac{\tilde N}{g_0V}\right)^{3/2},
\label{equ18.62}
\end{equation}
其中$T_c$被称为Bose凝聚~(condensation)温度。{\it 对于高于$T_c$的温度``积分分析''可行。在低于$T_c$时会发生``Bose凝聚'',在基态轨道上会出现反常的粒子布居数。}

在\eqref{equ18.62}式中取原子质量$m$和表观粒子数密度$\tilde N_e/g_0V$为液态\ce{^4He}的情况,可以得到凝聚温度$(\simeq\SI{3}{\kelvin})$接近于超流以及其他一些非经典效应出现的温度(\SI{2.17}{\kelvin})。考虑到我们将\ce{^4He}处理成了无相互作用理想气体,这个结果可以算是符合的非常好了。

为求得基态轨道以及其他低激发轨道态上的布居数,重新来看关于总粒子数的式子
\begin{equation}
\tilde N_e=\sum_{{\mathbf k},m_s}\bar n(\varepsilon_{\mathbf k})=g_0\sum_{\mathbf k}\left[\mathrm e^{\beta(\varepsilon_{\mathbf k}-\mu)-1}\right]^{-1}
\end{equation}
容许的能量值为
\begin{equation}
\varepsilon_{n_x,n_y,n_z}=\frac{p^2}{2m}=\frac{h^2}{2m}\left(\frac{1}{\lambda_x^2}+\frac{1}{\lambda_y^2}+\frac{1}{\lambda_z^2}\right)=\frac{h^2}{8mV^{2/3}}\left(n_x^2+n_y^2+n_z^2\right),
\end{equation}
其中我们再一次引用了量子力学关于动量和波长的关系($p=h/\lambda$),假定边长为$V^{1/3}$的方盒子,要求每一条边``贴合''相应半波长的整数倍($\frac{1}{2}n_x\lambda_x=V^{1/3}$等等)。这些离散量子力学态的能量与我们在\ref{sec16.5}节用以导出轨道态密度函数的那个一样。基态能量在$n_x=n_y=n_z=1$时取得(一般我们以这个为能量零点)。第一激发态有两个$n$是一,还有一个是二——这是一个三重简并态。能级差为$\varepsilon_{211}-\varepsilon_{111}=6h^2/mV^{2/3}$。对于一个一升($V=\SI{e-3}{\cubic\meter}$)的容器,取质量为\ce{^4He}的质量$(\simeq\SI{6.6e-27}{\kilo\gram})$,第一激发态的能量(相对于基态能量)是
\begin{equation*}
\varepsilon_{211}-\varepsilon_{111}=6h^2/mV^{3/2}\simeq\SI{2.5e-37}{\joule},
\end{equation*}
或者
\begin{equation}
(\varepsilon_{211}-\varepsilon_{111})/k_\text{B}\simeq\SI{2e-14}{\kelvin}.
\label{equ18.65}
\end{equation}
可以看到离散态之间的能量差{\it 非常}接近——比任何合理的温度下$k_\text{B}T$要小得多。把求和用积分替代应当是相当自然的呀!

让我们再仔细检查检查当化学势从下面逼近$\varepsilon_{111}$时各个态的布居数。我们尤其感兴趣的是基态轨道的布居数相比气体的总粒子数可观时$\mu$的值。令$n_0$为基态轨道上的粒子数,那么$[\exp(\varepsilon_{111}-\mu)-1]^{-1}=n_0$。若$n_0\gg 1$,则$\beta(\varepsilon_{111}-\mu)\ll 1$,可以将指数函数展开到一次项,则$n_0\sim k_\text{B}T/(\varepsilon_{111}-\mu)$。从而基态轨道上的粒子数和系统总粒子数(例如$n_0\simeq 10^{22}$)可比,出现在$\beta(\varepsilon_{111}-\mu)\sim 10^{-22}$时。

那么,此时第一激发态上的布居数是多少?能量差$(\varepsilon_{111}-\mu)/k_\text{B}$是$\simeq\SI{e-21}{\kelvin}$(对于$T\simeq\SI{10}{\kelvin}$),而$(\varepsilon_{211}-\varepsilon_{111})/k_\text{B}\simeq\SI{1e-14}{\kelvin}$(\ref{equ18.65}式)。从而得到$n_{211}/n_0\simeq 10^{-7}$。更高的态上的布居数依然下降地非常快。

随着Bose气体的温度下降,其摩尔Gibbs势上升并接近于基态轨道能量。基态轨道上的布居数上升,在临界温度$T_c$下相对于Bose气体总粒子数而言相当可观。而其他任意单个态上的布居数相对而言都是可略的。

随着温度进一步下降,$\mu$和基态能量的距离没法比$\beta(\mu-\varepsilon_{111})=1/\tilde N\simeq 10^{-23}$更近(此时基态上呆着气体中所有$N$个粒子!)。这里基态屏蔽了其他所有态,使$\mu$无法和它们更接近,这些态各自都只有相当少的粒子布居。自然,剩下的粒子都呆在基态上。

对于Bose凝聚中的物理有了这个认识之后,修正之前的分析将是非常简单的工作。除去基态外的所有轨道态用对轨道态密度的积分就可以很好的表示了。而基态需要从精确求和中预先区分出来单独处理。

粒子数为
\begin{equation}
\tilde N = n_0+\tilde N_e,
\label{equ18.66}
\end{equation}
其中$n_0$是基态轨道上的粒子数
\begin{equation}
n_0=\left(\mathrm e^{-\beta\mu}+1\right)^{-1}=\frac{\xi}{1-\xi},
\label{equ18.67}
\end{equation}
另外$\tilde N_e$是``激发态''({\bf e}xcited state,即除去基态以外的全部轨道态)上的粒子数。``激发粒子''的数目$\tilde N_e$由\eqref{equ18.53}式%
\mpar{译注:原文数个引用公式都提前了一个编号,已更正。}%
给出。

关于能量的\eqref{equ18.58}式仍然是对的,因为零能轨道上的布居不贡献能量。从而{\it 对理论的全部修正包括将$\tilde N_e$重新诠释成激发态上的粒子数,以及附带的\eqref{equ18.66}和\eqref{equ18.67}式。}

等价的,在计算巨正则势时可以简单的把之前得到的结果加上去(\eqref{equ18.51}式),得到基本方程
\begin{equation}
\Psi=g_0k_\text{B}T\ln(1-\xi)-g_0k_\text{B}T\frac{V}{\lambda^3}F_{5/2}(\xi),
\label{equ18.68}
\end{equation}
自然,其中的$\xi$是逸度$\mathrm e^{\beta\mu}$。

利用\eqref{equ18.56}到\eqref{equ18.60}式和\eqref{equ18.66}到\eqref{equ18.67}式,我们可以计算关于Bose流体形形色色的各种可观测性质。这些性质总结在表\ref{tab18.2}中,并在图\ref{fig18.4}中简要画出。
\begin{table}
\caption{理想Bose流体的性质}
\label{tab18.2}
\begin{tabular}{l}
\toprule
基本方程\\
$\Psi=k_\text{B}\ln(1-\xi)-k_\text{B}T(V/\lambda^3_T)F_{5/2}(\xi),$\\
凝聚温度\\
$k_\text{B}T_c=\frac{2\pi\hbar^2}{m}\left(\frac{1}{2.612}\frac{\tilde N}{g_0V}\right)^{3/2},$\\
凝聚和激发的boson\\
$\tilde N=n_0+\tilde N_e,\quad n_0=\frac{\xi}{1-\xi},\quad \tilde N_e=\frac{V}{\lambda^3}F_{3/2}(\xi)$;\\
$T>T_c:\quad n_0\ll \tilde N,\quad \tilde N_e\simeq \tilde N=\frac{V}{\lambda_3}F_{3/2}(\xi)$;\\
$T<T_c:\quad n_0/\tilde N=1-\tilde N_e/\tilde N=\left(1-\frac{T}{T_c}\right)^{3/2}$.\\
能量
$T>T_c:\quad U=\frac{3}{2}\tilde N_ek_\text{B}T\frac{F_{5/2}(\xi)}{F_{3/2}(\xi)};$\\
$T<T_c:\quad U=\frac{3}{2}\tilde N_ek_\text{B}T\frac{F_{5/2}(\xi)}{F_{3/2}(\xi)}\left(\frac{T}{T_c}\right)^{3/2}=0.76\tilde Nk_\text{B}T_c\left(\frac{T}{T_c}\right)^{5/2}.$\\
热容$c_v$(每粒子)\\
$T>T_c:\quad c_v=\frac{3}{2}k_\text{B}\left[\frac{5F_{5/2}(\xi)}{2F_{3/2}(\xi)}-\frac{3F'_{5/2}(\xi)}{2F'_{3/2}(\xi)}\right]$;\\
$T<T_c:\quad c_v=1.9k_\text{B}\left(\frac{T}{T_c}\right)^{3/2}$.\\
熵\\
$T>T_c:\quad S=\frac{5}{2}k_\text{B}\frac{V}{\lambda_T^3}F_{5/2}(\xi)-\tilde Nk_\text{B}\ln\xi$;\\
$T<T_c:\quad S=\frac{5}{2}l_\text{B}\frac{V}{\lambda_T^3}F_{5/2}(1)=3.35k_\text{B}\frac{V}{\lambda_T^3}$.\\
\bottomrule
\end{tabular}
\end{table}

\begin{figure}
\includegraphics[width=\textwidth]{Pictures/fig18.4.png}
\caption{理想Bose流体的性质。关于$T>T_c$处的能量和热容是随便画的。}
\label{fig18.4}
\end{figure}

首先,考虑基态粒子数的温度依赖。对于$T<T_c$呆在激发态的最大粒子数为
\begin{equation}
\tilde N_e=\frac{g_0V}{\lambda_T^3}F_{3/2}(1),\quad T<T_c,
\end{equation}
特别的,当$T\rightarrow T_c$,有$\tilde N_e\rightarrow \tilde N$,故
\begin{equation}
\tilde N=\frac{g_0V}{\lambda_T^3}F_{3/2}(1),
\end{equation}
其中$\lambda_c$是$\lambda_T$在$T=T_c$时的值。作除法
\begin{equation}
\frac{\tilde N_e}{\tilde N}=\left(\frac{\lambda_c}{\lambda_T}\right)^3=\left(\frac{T}{T_c}\right)^{3/2}.
\end{equation}
基态上的粒子数为
\begin{equation}
\frac{n_0}{\tilde N}=1-\frac{\tilde N_e}{\tilde N}=1-\left(\frac{T}{T_c}\right)^{3/2}.
\end{equation}
这个依赖关系在图\ref{fig18.4}中大致画出。

由于系统能量的导数就是热容这个易于观测的量,所以我们也对它挺感兴趣。$T>T_c$的情况下能量由\eqref{equ18.60}式给出。而对于$T<T_c$,可以将\eqref{equ18.58}式改写成
\begin{equation}
\begin{aligned}
U &=\frac{3}{2}k_\text{B}T\frac{g_0V}{\lambda_T^3}F_{3/2}(1)=\frac{3}{2}k_\text{B}T\frac{\tilde N_e}{F_{3/2}(1)}F_{5/2}(1), \\
 &=\frac{3}{2}\tilde Nk_\text{B}T\frac{F_{5/2}(1)}{F_{3/2}(1)}\frac{\tilde N_e}{\tilde N}=\frac{3}{2}\tilde Nk_\text{B}T(0.51)\left(\frac{T}{T_c}\right)^{3/2} \\
 &=0.76\tilde Nk_\text{B}T_c\left(\frac{T}{T_c}\right)^{5/2},\quad T<T_c.
\end{aligned}
\label{equ18.73}
\end{equation}

对于$T>T_c$,能量由\eqref{equ18.60}式给出,即$U=\frac{3}{2}\tilde N_ek_\text{B}T[F_{5/2}(\xi)/F_{3/2}(\xi)]$,故能量总是比其经典对应值要小。逸度关于$T$的关系可以从图\ref{fig18.2}中得到。

计算$T<T_c$下的摩尔热容可以直接通过\eqref{equ18.73}式求导得到
\begin{equation}
c_v=1.9\tilde Nk_\text{B}\left(\frac{T}{T_c}\right)^{3/2},\quad T<T_c.
\end{equation}
尤其值得主义得是,在$T=T_c$下$c_v=1.9Nk_\text{B}$,明显比高温下的经典结果$1.5Nk_\text{B}$要高。

计算$T>T_c$下的热容需要在$\tilde N$为常数的情况下对\eqref{equ18.60}式求导,并通过\eqref{equ18.56}式求得$(\mathrm d\xi/\mathrm dT)_V$。这个结果在图\ref{fig18.4}中大致画出,并在表\ref{tab18.2}中给出。

热容在$T=T_c$处的尖角是Bose凝聚的信号。类似的不连续性也在\ce{^4He}流体上观察到了;其具体的形状与两序参量的普遍性分类的重整化群理论的预言相符(复习第\ref{chap12}章的倒数第二段)。

最后我们注意\ce{^4He}的Bose凝聚伴随着一些新奇的现象。在$T_c$以下,流体能在毛细管中自由流动。它能从水盆边缘爬出去。正如它名字所体现的,``超流体''。对这些性质的解释超出了统计力学的范围。我们只需要知道它是``凝聚相'',或者说能在狭窄的管子中自由流动的{\it 基态分量}就好。这个分量很难通过摩擦耗散能量,因为它已经呆在基态上了。更具有意义的是,凝聚相具有{\it 量子相干性},这个概念没有经典对应物;所有的boson都呆在单个态上,关联在一起,与呆在激发态上的粒子(随机分布在大量的态上)完全不一样。

类似的Bose凝聚发生在特定金属的电子流体中。通过与声子相互作用,一对电子可以耦合在一起。电子对的行为如同boson一样。其Bose凝聚指向了超导电性,类似于\ce{^4He}中的超流性。

\noindent{\bf 习题}
\begin{itemize}
\item[18.7-1] 证明分别关于$\tilde N_e$和$U$的\eqref{equ18.56}式和\eqref{equ18.58}式在经典区域会趋向于相应的经典极限。
\item[18.7-2] 证明$F_{3/2}(1),F_{5/2}(1)$以及$F_{5/2}'(1)$都是有限的,而$F'_{3/2}(1)$是无穷大。这里$F'_{3/2}(1)$代表$F_{3/2}(x)$在$x=1$处的导数值。
\item[18.7-3] 证明精确考虑了基态轨道的贡献之后,会给巨正则配分求和加上一项$g_0k_\text{B}T\ln(1-\xi)$,从而验证\eqref{equ18.68}式。
\end{itemize}