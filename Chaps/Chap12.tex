%---------------------------------
%   Translator: SI
%   Proofreaders: 未校对
%---------------------------------
\chapter{宇宙学}
\label{chap12}

\section{宇宙学是什么?}
\label{sec12.1}
\subsection*{宇宙! in \ding{73} the \ding{73} large \ding{73} }
宇宙学 (cosmology) 研究整个宇宙:宇宙的历史、演化、成分、动力学。宇宙学的首要研究目的是理解宇宙的大尺度结构,但也为宇宙从大爆炸 (the Big Bang) 开始膨胀以来所有更小尺度的结构——星系、恒星、行星、人类——提供了背景与出发点。因此,宇宙学与其他学科(例如天文学的其它分支、物理学、生物学)的交叉是一片富饶的科研领域。此外,随着天文学家能够详细研究大爆炸的证据,宇宙学从而可以解释一些非常基本的物理问题:在可能范围内最大能量下的物理规律是什么、大爆炸是怎样发生的、大爆炸之前是什么、构成物质的基本成分(如电子质子中子)是怎样产生的。最后,自然界中所有系统与结构的起源都可以追溯到宇宙学的某些方面,甚至统治自然界的物理定律的起源也有可能这样。

人们对宇宙大尺度结构的理解,根本上依赖于广义相对论。这一点不难说明。粗略地说,牛顿理论在系统质量$M$与其尺度$R$相比很小(即$M/R \ll 1$)的时候可以很好地描述引力。当系统的$M/R$大到接近1的时候,必须用GR代替牛顿理论。后者对应的一种情况是,系统的半径$R$变小的速度大于质量$M$变小的速度,也就是致密或塌缩的系统:中子星、黑洞的半径与它们的质量相比十分小。另一种情况是,系统质量的增长速度比半径的增长速度快——这就是宇宙学的情形:如果空间各处的质量密度大致是均匀的,那么随着我们考虑$R$越来越大的系统,质量按$R^3$增长,而$M/R$因而越来越大,必须采用GR处理。

大到什么尺度必须要用GR呢?我们从太阳的中心出发考虑半径$R$越来越大的球状系统。太阳是非相对论性的,一旦$R$大于$R_\odot$,则$M$随着半径的增长几乎不变——直到半径增大到遇见下一个恒星。太阳的上级系统——星系,一般含有$10^{11}$个恒星,半径量级为$15\,\text{kpc}$。(1pc,也就是1秒差距(parsec, 缩写为pc),等于$3 \times 10^{16}\,\text{m}$。)星系的$M/R \sim 10^{-6}$,和太阳差不多。因此星系动力学也不需要相对论。(这只对星系整体成立,星系中心附近的小区域可能由黑洞或其它相对论性天体主导。) 人们观测到星系组成了星系团,星系团通常在$\text{Mpc}$尺度的区域内含有数千个星系,星系团的$M/R \sim 10^{-4}$,仍然不需要相对论就可很好地描述。

然而,当考虑比典型星系团更大的尺度时,我们就进入了\textbf{宇宙学}的领域。

在宇宙学的图景中,星系甚至星系团都是非常小尺度的结构,就像组成宽广宇宙的原子。人类的望远镜可以看到比$10 \,\text{Gpc}$更远的距离。在这种大尺度上观测到的宇宙是\textbf{均匀的 (homogeneous)},各处的星系密度大致相同,星系类型几乎一致。后面会看到,平均的质量-能量密度大致为$\rho = 10^{-26} \,\text{kg}\,\mathrm{m}^{-3}$。这一密度下,质量$M = 4\pi \rho R^3 / 3$与$R$相等的尺度为$R \sim 6\,\text{Gpc}$,在可观测宇宙的范围内。因此,为了理解望远镜观测到的宇宙,需要用到广义相对论。

实际上,广相是科学家用来研究宇宙学的第一个自洽的理论框架。存在着合适度规来描述人们观测到的均匀宇宙:这个宇宙没有边界 (boundry)、没有边缘 (edge)、处处均匀。牛顿引力无法自洽地描述这种模型,因为牛顿理论的基本方程$\nabla^2 \Phi = 4\pi G \rho$的解在没有边界、也就没有微分方程边界条件的情况下是糊涂的。因此只有在爱因斯坦理论的框架下,宇宙学才能成为物理学与天文学的分支。

下面考虑逆向的问题:在一个全体结构都是高度相对论性的宇宙中:
\begin{itemize}
\item 研究宇宙的局域部分可以不考虑宇宙学吗?
\item 在前几章应用广义相对论研究中子星和黑洞的时候,可以假设它们嵌在空的渐近平直时空当中,如果它们在高度相对论性的宇宙里,还可以像前几章那样假设吗?
\item 天文学家研究单个恒星、地质学家研究单个行星、生物学家研究单个细胞——都可以不考虑广相吗?
\end{itemize}
答案是“可以”,因为广相的时空是局域平直的:只要实验限制在局域范围内,就不需要考虑大尺度的几何。在牛顿理论中,这种整体与局域分离的性质不成立。根据牛顿理论,一个大尺度的、密度均匀的系统,即使是其局部引力场也依赖于远处的边界条件,依赖于遥远的宇宙“边缘”的形状(见本章习题3)。因此广相不仅使我们可以研究宇宙学,并且它还解释了为啥其它学科可以不考虑广相!

\subsection*{宇宙  arena \ding{73}}
近年来,随着地面与空间天文观测的进步,宇宙学已经成为了一门精确科学,可以回答一些最基本的物理问题。在大尺度——平均在大于$10\, \text{Mpc}$的尺度上观测到的宇宙基本图景是相当简单的:均匀的宇宙以处处相同的速率膨胀。宇宙也是\textbf{各向同性的 (isotropic)}:人们在各个方向观测到的宇宙平均而言都相同。宇宙布满了具有黑体热谱的辐射(温度为$2.725 \,\text{K}$)。宇宙在膨胀意味着宇宙的年龄有限,至少宇宙是从密度非常高的状态经有限长的时间膨胀来的。热辐射意味着宇宙早期比现在热得多,宇宙随着膨胀而越来越冷。膨胀解决了最古老的宇宙学难题——Olbers佯谬。夜空之所以是黑的,是因为我们不会接收到无限大的均匀宇宙中所有恒星发射的光,而只会收到距离足够近、发出的光可以在宇宙年龄之间到达地球的恒星。

不过,膨胀又引起了其它深刻的问题:
\begin{itemize}
\item 宇宙是怎样演化到当前状态的?更早时候的宇宙是什么样?
\item 第一颗恒星是如何形成的,为啥恒星成团组成星系,为啥星系也成团?小于$10 \,\text{Mpc}$的尺度上有各种各样的宇宙结构,造成这些结构的密度涨落是怎样产生的?
\item 元素是怎样形成的,在高温高密度、一般的原子核无法存在的早期,宇宙是什么样的?非常热的早期宇宙可以告诉我们一些更高能标(比粒子加速器还高)的物理学吗?
\item 观测到的宇宙均匀性与各向同性能否用物理解释?
\end{itemize}

对这些问题的研究促使物理学家探索一些基础物理学前沿的深刻问题。均匀性问题可以通过假设早期宇宙以指数型的速度快速膨胀(物理学称之为\textbf{暴胀(inflation)})加以解决。如果实验室可达到的高能标的物理规律有适当的形式,那么这个假设就成立,如果它成立,那么还能额外解释形成星系与星系团的密度涨落。后面会看到,宇宙中的大部分物质似乎都是未知的,物理学家称之为\textbf{暗物质 (dark matter)},因为它不发光(电磁辐射)。更奇特的是,宇宙似乎弥漫着一种相对论性的能量密度,它具有负压强,使宇宙膨胀的越来越快;物理学家称之为\textbf{暗能量 (dark energy)}。暗能量与暴胀之谜也许只有在更好地研究高能物理规律之后才能解决,因此理论物理学家正越来越盼望着从天文学中找到发展理论的线索。

现代宇宙学早已回答了这些问题中的一部分,并且这些答案越来越精确、明晰。本章是当前(2008年)对基本问题的理解的快照。与本书涉及的其他领域不同,未来宇宙学的研究很可能有着新见解(insights)、新惊奇(surprises)、甚至是一场新革命(revolution)。

\section{宇宙运动学:观测膨胀宇宙}
\label{sec12.2}