\chapter{临界现象}
\label{chap10}

\section{临界点附近的热力学}
\label{sec10.1}
前面几章讲述了热力学的整体结构,它在上世纪中叶就被证明是逻辑完整的,但是这个宏伟结构在一个看起来次要的细节之处失败了。这个细节正是系统在临界点附近的性质。经典热力学正确预言了各种“广义响应率”(generalized susceptibilities)(例如热容、压缩率、磁化率)在临界点处有分岔,并且强烈暗示了分岔的解析形式(或者“形状”)。确实有分岔,但是其解析形式不符合热力学预言,不符之处具有的规律性表明了存在经典热力学无法解释的潜在、独立原理。

对临界点附近巨大涨落的实验观测可以追溯到1869年,T. Andrews\footnote{T. Andrews, {\it Phil. Trans. Royal Soc. } {\bf 159}, 575(1869)}记录到流体的“临界乳光”(critical opalescence)现象。水在临界点附近($T = \SI{647.29}{K}, P = \SI{22.09}{MPa}$)的密度有巨大涨落,由此造成的光散射让水变成“乳白色”并且不透明。把该状态的水加热或冷却零点几K就能让它回到普通的透明状态。

类似的,磁系统的磁导率在临界点附近分岔,磁矩