\chapter{曲率的序言}
\label{chap5}

\begin{align}
    1\\2
\end{align}

\section{引力与曲率的关系}
\label{sec5.1}
目前我们都是在狭义相对论(SR)中讨论问题。力在SR当中的地位很重要,但是前面从来没有直接研究引力。SR的一个重要基础是存在覆盖整个时空的惯性系:全体时空可以由一个坐标系描述,这个系的所有坐标点都相对于原点静止,所有的坐标钟与原点的钟走时率相同。从这个基本假设可以导出时间间隔$\Delta s^2$的概念,它给物理事件赋予了具有不变性的几何意义。例如,两事件之间的类时间隔是经过这两个时件的钟所走过的时间;类空间隔是在这两个事件同时的坐标系当中的空间距离。

(未完成)

\section{极坐标系的张量代数}
\label{sec5.2}
考虑欧几里得平面。直角坐标$\{x, y\}$,极坐标$\{r, \theta\}$之间的关系为:
\begin{equation}
    \left.
    \begin{split}
    r &= (x^2 + y^2)^{1/2}, \quad x = r \cos \theta, \\
    \theta &= \Arctan (y / x), \quad y = r \sin \theta.
    \end{split}
    \right\}
\label{equ5.3}
\end{equation}

由直角坐标的微小增量$\Delta x, \Delta y$造成的$\Delta r, \Delta \theta$为
\begin{equation}
\left.
\begin{split}
    \Delta r &= \frac{x}{r} \Delta x + \frac{y}{r} \Delta y = \cos \theta \Delta x + \sin \theta \Delta y, \\
    \Delta \theta &= -\frac{y}{r^2} \Delta x + \frac{x}{r^2} \Delta y = -\frac{1}{r} \sin \theta \Delta x + \frac{1}{r} \cos \theta \Delta y,
\end{split}
\right\}
\label{equ5.4}
\end{equation}
上式到一阶小量都成立。

也可以使用其它坐标系。记一般的坐标系为$\{ \xi, \eta \}$:
\begin{equation}
\left.
\begin{split}
    \xi &= \xi (x, y), \quad \Delta \xi = \frac{\partial \xi}{\partial x} \Delta x + \frac{\partial \xi}{\partial y} \Delta y, \\
    \eta &= \eta (x, y), \quad \Delta \eta = \frac{\partial \eta}{\partial x} \Delta x + \frac{\partial \eta}{\partial y} \Delta y.
\end{split}
\right\}
\label{equ5.5}
\end{equation}
为了保证$(\xi, \eta)$是个好坐标系,任意两个不同的点$(x_1, y_1)$和$(x_2, y_2)$应该对应不同的$(\xi_1, \eta_1)$与$(\xi_2, \eta_2)$(通过\eqref{equ5.5}式对应)。例如,按$\xi = x, \eta = 1$定义的坐标系不是好坐标系,因为不同的两点$(x = 1, y = 2)$和$(x = 1, y = 3)$都对应$(\xi = 1, \eta = 1)$。数学上,这要求如果方程\eqref{equ5.5}中的$\Delta \xi = \Delta \eta = 0$,则必须对应相同的点,即$\Delta x = \Delta y = 0$. 这意味着\eqref{equ5.5}式的行列式非零:
\begin{equation}
    \det \begin{pmatrix}
        \partial \xi / \partial x & \partial \xi / \partial y \\
        \partial \eta / \partial x & \partial \eta / \partial y
    \end{pmatrix} 
    \neq 0.
\label{equ5.6}
\end{equation}
这个行列式称为坐标系变换\eqref{equ5.5}式的\textit{Jacobian}(\textit{雅可比行列式})。如果Jacobian在某一点为零,则称坐标变换在该点具有\textit{奇性 (singular)}。

\subsection*{向量与1形式}
向量的旧的定义是在\textit{任意}坐标变换下与位移的变换方式相同的量。也就是说,向量$\Delta \vec{r}$可以表示为\footnote{欧几里得空间的向量用箭头标记,其分量指标(1, 2)用希腊字母表示,求和对所有指标进行。}位移$(\Delta x, \Delta y)$,或者在极坐标系表示为$(\Delta r, \Delta \theta)$,或者在一般坐标系中为$(\Delta \xi, \Delta \eta)$. 根据\eqref{equ5.5}式,对于微小的$(\Delta x, \Delta y)$有:
\begin{equation}
    \begin{pmatrix}
        \Delta \xi \\ \Delta \eta
    \end{pmatrix}
    =
    \begin{pmatrix}
        \partial \xi / \partial x & \partial \xi / \partial y \\
        \partial \eta / \partial x & \partial \eta / \partial y 
    \end{pmatrix}
    \begin{pmatrix}
        \Delta x \\ \Delta y 
    \end{pmatrix}.
\label{equ5.7}
\end{equation}
定义变换矩阵
\begin{equation}
    (\Lambda\indices{^{\alpha'}_{\beta}}) = 
    \begin{pmatrix}
        \partial \xi / \partial x & \partial \xi / \partial y \\
        \partial \eta / \partial x & \partial \eta / \partial y 
    \end{pmatrix},
\label{equ5.8}
\end{equation}
则可以将任意向量$\vec{V}$的分量的变换规律写成与SR相同的形式:
\begin{equation}
    V^{\alpha'} = \Lambda\indices{^{\alpha'}_{\beta}} V^\beta,
\label{equ5.9}
\end{equation}
其中不带撇的指标代表$(x, y)$,带撇指标表示$(\xi, \eta)$,指标取$1, 2$. 向量可以定义为分量按照\eqref{equ5.9}式变换的这样的量。不过,存在一种更加复杂而自然的、现代的定义方式,下面进行介绍。

考虑平面上的标量场$\phi$。给定坐标系$(\xi, \eta)$就能计算偏导数$\partial \phi / \partial \xi$和$\partial \phi / \partial \eta$. \textit{定义}1形式$\trd \phi$为(在坐标系$(\xi, \eta)$中)具有如下分量的几何对象:
\begin{equation}
    \trd \phi \to (\partial \phi / \partial \xi, \partial \phi / \partial \eta).
\label{equ5.10}
\end{equation}
这是1形式的一般定义,每个标量场都定义了一个1形式。1形式分量的变换规律可以通过链式法则(chain rule)导出:
\begin{equation}
    \frac{\partial \phi}{\partial \xi} = \frac{\partial x}{\partial \xi} \frac{\partial \phi}{\partial x} + \frac{\partial y}{\partial \xi} \frac{\partial \phi}{\partial y},
\label{equ5.11}
\end{equation}
$\partial \phi / \partial \eta$同理。用\textit{行向量}可以方便地用矩阵形式表示:
\begin{equation}
\begin{pmatrix}
    \partial \phi / \partial \xi & \partial \phi / \partial \eta
\end{pmatrix}
=
\begin{pmatrix}
    \partial \phi / \partial x & \partial \phi / \partial y
\end{pmatrix}
\begin{pmatrix}
    \partial x / \partial \xi & \partial x / \partial \eta \\
    \partial y / \partial \xi & \partial y / \partial \eta
\end{pmatrix},
\label{equ5.12}
\end{equation}

1形式的变换矩阵可类比\eqref{equ5.8}式定义为一组$(x, y)$坐标关于$(\xi, \eta)$坐标的偏导数:
\begin{equation}
    ( \Lambda\indices{^\alpha_{\beta'}} ) = 
    \begin{pmatrix}
        \partial x / \partial \xi & \partial x / \partial \eta \\
        \partial y / \partial \xi & \partial y / \partial \eta
    \end{pmatrix}
\label{equ5.13}
\end{equation}
这样,\eqref{equ5.12}式就可以写成分量求和的形式:
\begin{equation}
    (\trd \phi)_{\beta'} = \Lambda\indices{^\alpha_{\beta'}} (\trd \phi)_\alpha.
\label{equ5.14}
\end{equation}
注意,上式的求和是对变换矩阵的\textit{第一个}分量进行的,这对应于行向量左乘矩阵。

值得一提的是,SR从来没考虑过行向量,因为Lorentz变换矩阵是个简单的对称矩阵。不过即使是上面的简单情况也要用到行向量。当张量的指标多于两个时,矩阵表示就非常累赘。GR需要处理四个甚至五个指标的张量,因此后面一般用代数形式(如\eqref{equ5.14}式)表示变换,后面就不再使用矩阵表示了。

本节已经看到,在现代观点之下,张量代数的基础是1形式的定义。它比旧定义更加自然,旧定义首先定义了\textit{单个}向量$(\Delta x, \Delta y)$,其它向量类比它定义。而现代定义利用偏导数定义了\textit{一类}1形式,1形式分量的变换规律自然地随之导出。

向量定义为将1形式映射为实数的线性函数。这个定义的具体含义在下一小节讲述。先来说明与SR形式的相似性,向量的变换规律为$\eqref{equ5.9}$式,有趣的是变换矩阵$(\Lambda\indices{^{\alpha'}_\beta})$和$( \Lambda\indices{^\alpha_{\beta'}} )$互逆。它们相乘得到:
\begin{align}
    \begin{pmatrix}
        \partial \xi / \partial x & \partial \xi / \partial y \\
        \partial \eta / \partial x & \partial \eta / \partial y 
    \end{pmatrix}
    \begin{pmatrix}
        \partial x / \partial \xi & \partial x / \partial \eta \\
        \partial y / \partial \xi & \partial y / \partial \eta
    \end{pmatrix} \notag \\
    =
    \begin{pmatrix}
        \dfrac{\partial \xi}{\partial x} \dfrac{\partial x}{\partial \xi} + \dfrac{\partial \xi}{\partial y} \dfrac{\partial y}{\partial \xi} & \dfrac{\partial \xi}{\partial x} \dfrac{\partial x}{\partial \eta} + \dfrac{\partial \xi}{\partial y} \dfrac{\partial y}{\partial \eta}  \\
        \dfrac{\partial \eta}{\partial x} \dfrac{\partial x}{\partial \xi} + \dfrac{\partial \eta}{\partial y} \dfrac{\partial y}{\partial \xi} & \dfrac{\partial \eta}{\partial x} \dfrac{\partial x}{\partial \eta} + \dfrac{\partial \eta}{\partial y} \dfrac{\partial y}{\partial \eta}
    \end{pmatrix}.
\label{equ5.15}
\end{align}
利用链式法则与偏导数的定义可以算出结果为
\begin{equation}
    \begin{pmatrix}
        \partial \xi / \partial \xi & \partial \xi / \partial \eta \\
        \partial \eta / \partial \xi & \partial \eta / \partial \eta
    \end{pmatrix}
    = 
    \begin{pmatrix}
        1 & 0 \\
        0 & 1
    \end{pmatrix}。
\label{equ5.16}
\end{equation}

\subsection*{曲线与向量}
通常所说的曲线是平面上一系列连续的点,我们把它称作\textit{路径 (path)},而把曲线专指参数化的路径。这也是现代数学的做法,将\textit{曲线 (curve)}定义为从实数区间到平面路径的映射。这意味着曲线是每一点都对应一个实数的路径,实数称为参数(parameter),记作$s$。每一点的坐标表示为参数$s$的函数就定义了平面上的一条曲线:
\begin{shaded}
\begin{equation}
    \text{曲线:} \{ \xi = f(s), \eta = g(s), \quad a \le s \le b \}
\label{equ5.17}
\end{equation}
\end{shaded}
将参数改变为$s' = s'(s)$(新参数是旧参数的函数,点不变),则有
\begin{equation}
    \text{曲线:} \{ \xi = f'(s'), \eta = g'(s'), \quad a' \le s' \le b' \},
\label{equ5.18}
\end{equation}
其中$f', g'$是\textit{新的}函数,而$a' = s'(a), b' = s'(b)$. 上式在数学上是一条\textit{新的}曲线,尽管它的\textit{像 (image)}(所经过的平面上的点)与原来相同。因此同一路径对应无数条曲线。

标量场$\phi$沿曲线的导数是$\rd \phi / \rd s$,依赖于$s$,因此变换参数,导数也随之变换。可以将导数写为
\begin{equation}
    \frac{\rd \phi}{\rd s} = \langle \trd \phi, \vec{V} \rangle,
\label{equ5.19}
\end{equation}
其中$\vec{V}$是分量为$(\rd \xi / \rd s, \rd \eta / \rd s)$的向量,这个向量只与曲线有关,而$\trd \phi$只依赖于$\phi$. 因此$\vec{V}$是与曲线特征有关的向量,称之为\textit{切向量 (tangent vector)}。(见图\ref{fig5.4},显然它与曲线相切) \ \sout{画外音:哪里显然了……}

所以,向量可以看作是给定$\phi$而产生$\rd \phi / \rd s$的东西。这就引出了最现代的观点,曲线的切向量应该\textit{称为} $\rd / \rd s$。有些相对论文献偶尔使用这一符号。不过我们把它记作$\vec{V}$,知道它的分量是$(\rd \xi / \rd s, \rd \eta / \rd s)$就好了。注意,平面上的一条路径上的任一点都有着无数切向量,它们的方向相同而长度不同,它们可以视为\textit{不同}曲线(在一点邻域中的参数化不同)的切向量。曲线是给定了参数的路径,因此曲线的切向量\textit{唯一}。此外,即使两条曲线在某一点的切向量相等,它们在其它点也可以不同,根据Taylor展开式$\xi (s + 1) \approx \xi (s) + \rd \xi / \rd s$可见,$\vec{V} (s)$近似沿曲线从$s$到$s + 1$延展。

{
    \centering
    \includegraphics[width=0.6\textwidth]{fig5.4.png}
    \figcaption{\textit{一条曲线,曲线的参数,以及曲线的切向量}}
    \label{fig5.4}
}

注意$s$在坐标变换下不变(它的定义和坐标系无关),而$\vec{V}$的分量变化,因此根据链式法则可得
\begin{equation}
    \begin{pmatrix}
        \rd \xi / \rd s \\
        \rd \eta / \rd s
    \end{pmatrix}
    =
    \begin{pmatrix}
        \partial \xi / \partial x & \partial \xi / \partial y \\
        \partial \eta / \partial x & \partial \eta / \partial y
    \end{pmatrix}
    \begin{pmatrix}
        \rd x / \rd s \\
        \rd y / \rd s
    \end{pmatrix}.
\label{equ5.20}
\end{equation}
这与前面的向量变换律\eqref{equ5.7}式相同。

总结一下现代观点,向量是与某条曲线相切、将$\trd \phi$映射为$\rd \phi / \rd s$的线性函数。这样,下面就能更进一步地研究极坐标系。

\subsection*{极坐标系的1形式基与向量基}
显然,坐标基向量的变换规律为:
\begin{equation*}
    \vec{e}_{\alpha'} = \Lambda\indices{^\beta_{\alpha'}} \vec{e}_\beta,
\end{equation*}
在极坐标下:
\begin{align}
    \vec{e}_r &= \Lambda\indices{^x_r} \Ve_x + \Lambda\indices{^y_r} \Ve_y \label{equ5.21} \\
    &= \frac{\partial x}{\partial r} \Ve_x + \frac{\partial y}{\partial r} \Ve_y \notag \\
    &= \cos \theta \Ve_x + \sin \theta \Ve_y, \label{equ5.22}
\end{align}
类似有
\begin{align}
    \Ve_\theta &= \frac{\partial x}{\partial \theta} \Ve_x + \frac{\partial y}{\partial \theta} \Ve_y \notag \\
    &= -r \sin \theta \Ve_x + r \cos \theta \Ve_y, \label{equ5.23}
\end{align}
注意,上式已经利用了
\begin{equation}
    \Lambda\indices{^x_r} = \frac{\partial x}{\partial r}.
\label{equ5.24}
\end{equation}
类似地,“反向”变换的矩阵元为
\begin{equation}
    \Lambda\indices{^r_x} = \frac{\partial r}{\partial x}.
\label{equ5.25}
\end{equation}
这个变换矩阵十分简单:矩阵指标的上下顺序对应到求导的上下关系就好了。

1形式基的关系可类似求出:
\begin{align}
    \trd \theta &= \frac{\partial \theta}{\partial x} \trd x + \frac{\partial \theta}{\partial y} \trd y, \notag \\
    &= -\frac{1}{r} \sin \theta \trd x + \frac{1}{r} \cos \theta \trd y. \label{equ5.26}
\end{align}
(注意上式与普通的微积分运算\eqref{equ5.4}式相似)。同样可得
\begin{equation}
    \trd r = \cos \theta \trd x + \sin \theta \trd y.
\label{equ5.27}
\end{equation}
根据以上内容可以画出不同点的基(图\ref{fig5.5})。容易画出基向量,1形式基可以画出$\trd r$和$\trd \theta$的等$r$、等$\theta$面辅助进行,不同位置的面的指向不同。

{
    \centering
    \includegraphics[width=0.7\textwidth]{fig5.5.png}
    \figcaption{极坐标系的向量基与1形式基图示}
    \label{fig5.5}
}

上面体现了一个非常重要的事实:各点的基互不相同。例如,图\ref{fig5.5}中$A$点与$C$点的向量基不平行。这是由于基向量指向坐标增加的方向,而这个方向随着点的改变而改变。此外,基的长度也不是恒定不变的。例如,根据\eqref{equ5.23}式可得

\begin{subequations}
\begin{alignat}{2}
    |\Ve_\theta|^2 &= \Ve_\theta \cdot \Ve_\theta = r^2 \sin^2 \theta + r^2 \cos^2 \theta = r^2,&& \label{equ5.28a} \\
    |\Ve_r| &= 1, \quad |\trd r| = 1, \quad |\trd \theta| = r^{-1}. &&\label{equ5.28b}
\end{alignat}
\end{subequations}
距离原点越远,$\Ve_\theta$的模长越大。因为$\Ve_\theta$在$(r, \theta)$系的分量为$(0, 1)$,意味着它表示$\theta$分量的1单位的位移,即1弧度。在半径更大的地方,移动1弧度的长度更大。因此极坐标基并非\textit{单位}基。其它基的模长容易求出。可以发现,$| \trd \theta|$在$r = 0$附近更大(更紧密),因为一个给定的向量在原点附近覆盖的$\theta$范围更大。

\subsection*{度规张量}
上面的点乘结果都是根据直角坐标系$x, y$中已知的度规来计算的:
\[
    \vec{e}_x \cdot \vec{e}_x = \vec{e}_y \cdot \vec{e}_y = 1, \quad \vec{e}_x \cdot \vec{e}_y = 0;
\]
或者用张量记号写为
\begin{equation}
    \mathbf{g} (\vec{e}_\alpha, \vec{e}_\beta) = \delta_{\alpha \beta} \quad \text{在直角坐标系中。}
\label{equ5.29}
\end{equation}
$\mathbf{g}$在极坐标系的分量是什么?根据分量的定义:
\begin{equation}
    g_{\alpha' \beta'} = \mathbf{g} (\vec{e}_{\alpha'}, \vec{e}_{\beta'}) = \vec{e}_{\alpha'} \cdot \vec{e}_{\beta'},
\label{equ5.30}
\end{equation}
或者根据\eqref{equ5.28}、\eqref{equ5.22}和\eqref{equ5.23}式可得
\begin{equation}
    g_{rr} = 1, \quad g_{\theta \theta} = r^2, \quad  g_{r \theta} = 0. \label{equ5.31}
\end{equation}
由此可得$\mathbf{g}$在极坐标系的分量为
\begin{equation}
    (g_{\alpha \beta})_{\text{polar}} = 
        \begin{pmatrix}
            1 & 0 \\
            0 & r^2
        \end{pmatrix},
\label{equ5.32}
\end{equation}
线元(line element)可以方便地同时表示$\mathbf{g}$的分量以及坐标,线元即为任意“无穷小”位移$\rd \vec{\ell}$的模:
\begin{shaded}
\begin{align}
    \rd \vec{\ell} \cdot \rd \vec{\ell} &= \rd s^2 = |\rd r \vec{e}_r + \rd \theta \vec{e}_\theta |^2 \notag \\
    &= \rd r^2 + r^2 \rd \theta^2. \label{equ5.33}
\end{align}
\end{shaded}
\textit{不要}将这里的$\rd r, \rd \theta$与1形式基$\trd r, \trd \theta$混淆,前者是$\rd \vec{\ell}$在极坐标系的分量,“$\rd$”就是“无穷小$\Delta$”的意思。 

另外有一种导出\eqref{equ5.33}式的方法值得一提。回顾\eqref{equ3.26}式,它表明任何$\binom{0}{2}$张量可以表示为$\binom{0}{2}$张量基$\trd x^\alpha \otimes \trd x^\beta$的线性组合:
\[
    \mathbf{g} = g_{\alpha \beta} \trd x^\alpha \otimes \trd x^\beta = \trd r \otimes \trd r + r^2 \trd \theta \otimes \trd \theta.
\]
尽管上式看起来像\eqref{equ5.33}式,但它们不一样:上式各项是算符,作用于向量$\rd \vec{\ell}$(其分量为$\rd r, \rd \theta$)之后得到\eqref{equ5.33}式。由于相应学科的符号混乱,导致上面两个式子非常不幸地十分相像。大多数教材与论文仍采用“旧式的”表达式——方程\eqref{equ5.33}来表示度规分量,本书遵从这一习惯。

度规分量矩阵存在逆矩阵:
\begin{equation}
    {\begin{pmatrix}
        1 & 0 \\
        0 & r^2
    \end{pmatrix} }^{-1} = 
    \begin{pmatrix}
        1 & 0 \\
        0 & r^{-2}
    \end{pmatrix}.
\label{equ5.34}
\end{equation}
由此可得$g^{rr} = 1, g^{r\theta} = 0, g^{\theta \theta} = 1/r^2$。这可以建立1形式与向量之间的映射。例如,向量场$\phi$的梯度场为$\trd \phi$,则这个1形式相应的向量$\rd \phi$具有分量
\begin{equation}
    (\vec{\rd} \phi)^\alpha = g^{\alpha \beta} \phi_{, \beta}
\label{equ5.35}
\end{equation}
或者写为
\begin{subequations}
\begin{alignat}{2}
    (\vec{\rd} \phi)^r &= g^{r \beta} \phi_{, \beta} = g^{rr} \phi_{, r} + g^{r \theta} \phi_{, \theta} \notag &&   \\
    &= \frac{\partial \phi}{\partial r}. &&\label{equ5.36a} \\
    (\vec{\rd} \phi)^\theta &= g^{\theta r} \phi_{, r} + g^{\theta \theta} \phi_{, \theta} && \notag \\
    &= \frac{1}{r^2} \frac{\partial \phi}{\partial \theta}. && \label{equ5.36b}
\end{alignat}
\end{subequations}
可见,1形式的分量是$(\phi_{, r}, \phi_{, \theta})$,而相应向量的分量是$(\phi_{, r}, \phi_{, \theta} / r^2)$。就算是在欧几里得空间,向量与相应的1形式的分量一般也不相同。直角坐标系是它们在其中唯一相等的坐标系。



\section{极坐标系的张量微积分}
\label{sec5.3}
极坐标系的基向量并非处处相等,这对向量求导产生了麻烦。例如,简单的直角坐标基$\vec{e}_x$,它在各点都相等,$\vec{e}_x$在极坐标系中的分量为$\vec{e}_x \to (\Lambda\indices{^r_x}, \Lambda\indices{^\theta_x}) = (\cos \theta, -r^{-1} \sin \theta)$。尽管$\vec{e}_x$处处相同,但显然它的分量不是常数,这是由于分量对应的基向量在各点不同。如果只把分量对坐标,例如对$\theta$求导,结果显然\textit{并非}$\partial \vec{e}_x / \partial \theta$,因为后者必须等于零。

从这个例子可见,对向量分量的求导结果一般情况下不是向量的导数,必须要考虑到基向量的变化。这是理解曲线坐标系与弯曲空间的关键。下面来系统讨论这些内容。

\subsection*{基向量的导数}
由于$\vec{e}_x$和$\vec{e}_y$是常向量场(在各点的值相同):
\begin{subequations}
\begin{alignat}{2}
    \frac{\partial}{\partial r} \vec{e}_r &= \frac{\partial}{\partial r} (\cos \theta \vec{e}_x + \sin \theta \vec{e}_y ) = 0,  &&\label{equ5.37a} \\
    \frac{\partial}{\partial \theta} \vec{e}_r &= \frac{\partial}{\partial \theta} (\cos \theta \vec{e}_x + \sin \theta \vec{e}_y) && \notag \\
    &= -\sin \theta \vec{e}_x + \cos \theta \vec{e}_y = \frac{1}{r} \vec{e}_\theta. && \label{equ5.37b}
\end{alignat}
\label{equ5.37}
\end{subequations}
这个结果有着简单的图像,如图\ref{fig5.6}。在相近的两点$A, B$,它们的$\vec{e}_r$的指向是从原点向外的径向,而有微小的差别。$\vec{e}_r$对$\theta$的导数就是$A, B$点的$\vec{e}_r$之差除以$\Delta \theta$。从图中可见,这个差平行于$\vec{e}_\theta$,这与方程\eqref{equ5.37b}相符。

{
    \centering
    \includegraphics[width=0.65\textwidth]{fig5.6.png}
    \figcaption{\textit{$\theta$变化$\Delta \theta$造成的$\vec{e}_r$的变化。}}
    \label{fig5.6}
}

同理可得
\begin{subequations}
\begin{alignat}{2}
    \frac{\partial}{\partial r} \vec{e}_\theta &= \frac{\partial}{\partial r} (-r \sin \theta \vec{e}_x + r \cos \theta \vec{e}_y ) = 0, && \notag \\ 
    &= -\sin \theta \vec{e}_x + \cos \theta \vec{e}_y = \frac{1}{r} \vec{e}_\theta,  && \label{equ5.38a} \\
    \frac{\partial}{\partial \theta} \vec{e}_\theta &= -r \cos \theta \vec{e}_x - r \sin \theta \vec{e}_y = -r \vec{e}_r. && \label{equ5.38b}
\end{alignat}
\label{equ5.38}
\end{subequations}
米娜桑可以画类似\ref{fig5.6}那样的图来解释这些求导结果。

\subsection*{一般向量的导数}
回到$\vec{e}_x$的导数,既然
\begin{equation}
    \vec{e}_x = \cos \theta \vec{e}_r - \frac{1}{r} \sin \theta \vec{e}_\theta, 
\label{equ5.39}
\end{equation}
则有
\begin{align}
    \frac{\partial}{\partial \theta} \vec{e}_x =& \frac{\partial}{\partial \theta} (\cos \theta) \vec{e}_r + \cos \theta \frac{\partial}{\partial \theta} (\vec{e}_r) \notag \\
    & -\frac{\partial}{\partial \theta} \left( \frac{1}{r} \sin \theta \right) \vec{e}_\theta - \frac{1}{r} \sin \theta \frac{\partial}{\partial \theta} (\vec{e}_\theta) \label{equ5.40} \\
    =& -\sin \theta \vec{e}_r + \cos \theta \left( \frac{1}{r} \vec{e}_\theta \right) \notag \\
    & - \frac{1}{r} \cos \theta \vec{e}_\theta - \frac{1}{r} \sin \theta (-r \vec{e}_r). \label{equ5.41}
\end{align}
其中利用了\eqref{equ5.37}, \eqref{equ5.38}式。化简上式可得
\begin{equation}
    \frac{\partial}{\partial \theta} \vec{e}_x = 0,
\label{equ5.42}
\end{equation}
正如所料。\eqref{equ5.40}式中的第一、三项是$\vec{e}_x$的极坐标系分量的求导结果,另外两项是对极坐标系基向量的求导结果,它们彼此相消结果为零。

一般的向量$\vec{V}$在极坐标系的分量$(V^r, V^\theta)$,向量的导数,类比\eqref{equ5.40}式,等于:
\begin{align*}
    \frac{\partial \vec{V}}{\partial r} &= \frac{\partial}{\partial r} (V^r \vec{e}_r + V^\theta \vec{e}_\theta) \\
    &= \frac{\partial V^r}{\partial r} \vec{e}_r + V^r \frac{\partial \vec{e}_r}{\partial r} + \frac{\partial V^\theta}{\partial r} \vec{e}_\theta + V^\theta \frac{\partial \vec{e}_\theta}{\partial r},
\end{align*}
$\partial \vec{V} / \partial \theta$类似。上式用指标记号写为
\[
    \frac{\partial \vec{V}}{\partial r} = \frac{\partial}{\partial r} (V^\alpha \vec{e}_\alpha) = \frac{\partial V^\alpha}{\partial r} \vec{e}_\alpha + V^\alpha \frac{\partial \vec{e}_\alpha}{\partial r}.
\]
其中指标$\alpha$跑遍$r, \theta$。

由此可见,$\vec{V}$的导数不仅包括其分量$V^\alpha$的导数。$r$是坐标之一,上式对$r$的导数可以推广为对一般坐标的导数:
\begin{equation}
    \frac{\partial \vec{V}}{\partial x^\beta} = \frac{\partial V^\alpha}{\partial x^\beta} \vec{e}_\alpha + V^\alpha \frac{\partial \vec{e}_\alpha}{\partial x^\beta},
\label{equ5.43}
\end{equation}
其中$x^\beta, \beta = 1, 2$分别对应$r, \theta$。

\subsection*{Christoffel符号}
方程\eqref{equ5.43}的最后一项显然十分重要。由于$\partial \vec{e}_\alpha / \partial x^\beta$本身是个向量,它也可以表示为基向量的线性组合;线性组合的系数记作$\Gamma\indices{^\mu_{\alpha \beta}}$:
\begin{equation}
    \frac{\partial \vec{e}_\alpha}{\partial x^\beta} = \Gamma\indices{^\mu_{\alpha \beta}} \vec{e}_\mu.
\label{equ5.44}
\end{equation}
$\Gamma\indices{^\mu_{\alpha \beta}}$的含义是向量$\partial \vec{e}_\alpha / \partial x^\beta$的$\mu$分量。它有三个分量:一个分量($\alpha$)表明对哪个基向量求导;第二个($\beta$)表明基向量对哪个坐标求导;第三个($\mu$)表示求导结果向量的分量。$\Gamma\indices{^\mu_{\alpha \beta}}$十分有用,值得给它进行命名:称之为\textbf{Christoffel 符号}。Christoffel符号是不是张量的分量?这个问题之后讨论。

前面实际上已经算出了极坐标系的Christoffel,从方程\eqref{equ5.37}和\eqref{equ5.38}可得
\begin{equation}
\left.
\begin{split}
    (1)\ \ \dfrac{\partial \vec{e}_r}{\partial r} &= 0  \Rightarrow \Gamma\indices{^\mu_{rr}} = 0, \quad \forall\, \mu, \\
    (2)\ \  \dfrac{\partial \vec{e}_r}{\partial \theta} &= \dfrac{1}{r} \vec{e}_\theta \Rightarrow  \Gamma\indices{^r_{r\theta}} = 0, \quad \Gamma\indices{^\theta_{r \theta}} = \dfrac{1}{r}, \\
    (3)\ \  \dfrac{\partial \vec{e}_\theta}{\partial r} &= \dfrac{1}{r} \vec{e}_\theta \Rightarrow \Gamma\indices{^r_{\theta r}} = 0, \quad \Gamma\indices{^\theta_{\theta r}} = \dfrac{1}{r}, \\
    (4)\ \  \dfrac{\partial \vec{e}_\theta}{\partial \theta} &= -r \vec{e}_r \Rightarrow \Gamma\indices{^r_{\theta \theta}} = -r, \quad \Gamma\indices{^\theta_{\theta \theta}} = 0.
\end{split}
\right\}
\label{equ5.45}
\end{equation}
定义式\eqref{equ5.44}当中的所有指标都必须在同一个坐标系中。因此,尽管我们利用了$\vec{e}_x, \vec{e}_y$是常向量场的性质导出了$\vec{e}_r, \vec{e}_\theta$的导数,但是\eqref{equ5.45}中的各方程并不依赖直角坐标系。Christoffel符号的重要性在于,它可以将极坐标系基向量的导数只用极坐标系的量表示,而与其它坐标系无关。

\subsection*{协变导数}
利用Christoffel符号的定义\eqref{equ5.44}式,可以将\eqref{equ5.43}式的导数表示为
\begin{equation}
    \frac{\partial \vec{V}}{\partial x^\beta} = \frac{\partial V^\alpha}{\partial x^\beta} \vec{e}_\alpha + V^\alpha \Gamma\indices{^\mu_{\alpha \beta}} \vec{e}_\mu.
\label{equ5.46}
\end{equation}
最后一项表示了对傀儡指标$\alpha, \mu$的两个求和,对这两个傀儡指标重新命名:$\mu$换成$\alpha$,$\alpha$换成$\mu$得到
\begin{equation}
    \frac{\partial \vec{V}}{\partial x^\beta} = \frac{\partial V^\alpha}{\partial x^\beta} \vec{e}_\alpha + V^\mu \Gamma\indices{^\alpha_{\mu \beta}} \vec{e}_\alpha.
\label{equ5.47}
\end{equation}
这样重命名了傀标之后上式就可提出等号右侧的$\vec{e}_\alpha$项:
\begin{equation}
    \frac{\partial \vec{V}}{\partial x^\beta} = \left( \frac{\partial V^\alpha}{\partial x^\beta} + V^\mu \Gamma\indices{^\alpha_{\mu \beta}} \right) \vec{e}_\alpha.
\label{equ5.48}
\end{equation}
因此,向量场$\partial \vec{V} / \partial x^\beta$的分量为
\begin{equation}
    \frac{\partial V^\alpha}{\partial x^\beta} + V^\mu \Gamma\indices{^\alpha_{\mu \beta}}.
\label{equ5.49}
\end{equation}
回顾一下偏导数的简写记号:$\partial V^\alpha / \partial x^\beta = V\indices{^\alpha_{, \beta}}$,利用这个记号并定义一个\textit{新}记号:
\begin{shaded}
\begin{equation}
    V\indices{^\alpha_{; \beta}} := V\indices{^\alpha_{, \beta}} + V^\mu \Gamma\indices{^\alpha_{\mu \beta}}.
\label{equ5.50}
\end{equation}
\end{shaded}
在这个分号简写记号下:
\begin{shaded}
\begin{equation}
    \frac{\partial \vec{V}}{\partial x^\beta} = V\indices{^\alpha_{; \beta}} \vec{e}_\alpha
\label{equ5.51}
\end{equation}
\end{shaded}
它将\eqref{equ5.48}式表示成了非常紧凑的形式。

若将$\beta$视为固定不动的,则$\partial \vec{V} / \partial x^\beta$是个向量场。但是实际上$\beta$\textit{可以}有两种取值,所以$\partial \vec{V} / \partial x^\beta$可以视为将向量$\vec{e}_\beta$映为向量$\partial \vec{V} / \partial x^\beta$的$\binom{1}{1}$张量场,就像第\ref{chap3}章习题17那样。这个张量场称为$\vec{V}$的\textit{协变导数 (covariant derivative)},(很自然地)记作$\nabla \vec{V}$,它的分量为
\begin{equation}
    (\nabla \vec{V})\indices{^\alpha_\beta} = (\nabla_\beta \vec{V})\indices{^\alpha} = V\indices{^\alpha_{; \beta}} 
\label{equ5.52}
\end{equation}
在直角坐标系中,这些分量就等于$V\indices{^\alpha_{, \beta}}$。然而在曲线坐标系中,必须考虑基向量的导数,我们得到了$V\indices{^\alpha_{; \beta}}$是$\nabla \vec{V}$的分量,不论在哪个曲线坐标系,只要把那个系的Christoffel符号带入\eqref{equ5.50}式就能算出。上述内容的重要性不可低估,它是之后所有内容的基础。存在一个叫做$\nabla \vec{V}$的$\binom{1}{1}$张量,它在直角坐标系中的分量是$\partial V^\alpha / \partial x^\beta$,在一般的坐标系$\{ x^{\mu'} \}$中的分量是$V\indices{^{\alpha'}_{; \beta'}}$,这个分量可以通过两种方式计算:
\begin{itemize}
\item 直接在$\{ x^{\mu'} \}$坐标系中利用\eqref{equ5.50}式和那个系的Christoffel符号$\Gamma\indices{^{\alpha'}_{\mu' \beta'}}$计算;
\item 利用从直角坐标系到一般坐标系$\{ x^{\mu'} \}$的张量分量变换律计算。
\end{itemize}

标量的协变导数是啥?协变导数与普通偏导数的不同是由于基向量随着坐标变化。因为标量不依赖于基向量,因此标量的协变导数等于普通的偏导数,也就是梯度:
\begin{equation}
    \nabla_\alpha f = \frac{\partial f}{\partial x^\alpha}; \quad \nabla f = \trd f.
\label{equ5.53}
\end{equation}


\subsection*{散度,Laplacian}
在深入讨论之前,我们来研究一下与之前的内容有联系的部分。在直角坐标系中,向量$V^\alpha$的散度是$V\indices{^\alpha_{, \alpha}}$,它是由$V\indices{^\alpha_{, \beta}}$对上下指标缩并产生的标量。缩并是不依赖于坐标系的运算,因此$\vec{V}$的散度也可以在坐标系$\{ x^{\mu'} \}$中对$\nabla \vec{V}$的两个指标进行缩并得到,其结果为标量$V\indices{^{\alpha'}_{; \alpha'}}$,注意,它与直角坐标系中的$V\indices{^\alpha_{, \alpha}}$是\textbf{同一个量}:
\begin{equation}
    V\indices{^\alpha_{, \alpha}} \equiv V\indices{^{\beta'}_{; \beta'}}
\label{equ5.54}
\end{equation}
其中不带撇的指标表示直角坐标系,带撇的表示任意坐标系。

极坐标系(简明起见,用不带撇的指标表示)中:
\begin{equation*}
    V\indices{^\alpha_{; \alpha}} = \frac{\partial V^\alpha}{\partial x^\alpha} + \Gamma\indices{^\alpha_{\mu \alpha}} V_{\mu}.
\end{equation*}
根据\eqref{equ5.45}式可以算出:
\begin{equation}
\left.
\begin{split}
    \Gamma\indices{^\alpha_{r \alpha}} =& \Gamma\indices{^r_{rr}} + \Gamma\indices{^\theta_{r \theta}} = \frac{1}{r}, \\
    \Gamma\indices{^\alpha_{\theta \alpha}} =& \Gamma\indices{^r_{\theta r}} + \Gamma\indices{^\theta_{\theta \theta}} = 0.
\end{split}
\right\}
\label{equ5.55}
\end{equation}
从而
\begin{align}
    V\indices{^\alpha_{; \alpha}} =& \frac{\partial V^r}{\partial r} + \frac{\partial V^\theta}{\partial \theta} + \frac{1}{r} V^r, \notag \\
    =& \frac{1}{r} \frac{\partial}{\partial r} (r V^r) + \frac{\partial}{\partial \theta} V^\theta. \label{equ5.56}
\end{align}
上式看起来很面熟。更面熟的是梯度的散度,也就是Laplacian。但是散度是对向量定义的,而梯度是个1形式,因此必须先把1形式转换为向量。给定一个标量场$\phi$,根据\ref{sec5.2}节最后的\eqref{equ5.53}式可得,相应的梯度向量的分量为$(\phi_{, r},\ \phi_{, \theta} / r^2$。将它带入向量的散度公式\eqref{equ5.56}可得
\begin{equation}
    \nabla \cdot \nabla \phi := \nabla^2 \phi = \frac{1}{r} \frac{\partial}{\partial r} \left( r \frac{\partial \phi}{\partial r} \right) + \frac{1}{r^2} \frac{\partial^2 \phi}{\partial \theta^2}.
\label{equ5.57}
\end{equation}
这就是平面极坐标系中的Laplacian的表达式,它等同于
\begin{equation}
    \nabla^2 \phi = \frac{\partial^2 \phi}{\partial x^2} + \frac{\partial^2 \phi}{\partial y^2}.
\label{equ5.58}
\end{equation}



\subsection*{1形式与高阶张量的导数}
标量场$\phi$不依赖于基向量,它的导数$\trd \phi$与协变导数$\nabla \phi$相同,今后都采用符号$\trd \phi$。为了计算1形式的导数(和向量的导数一样并非仅仅是对分量求偏导数),我们利用1形式作用于向量得到标量的性质。设$\tilde{p}$是1形式,$\vec{V}$是向量,对于固定的指标$\beta$,$\nabla_\beta \tilde{p}$也是一个1形式,$\nabla_\beta \vec{V}$是向量,而$\langle \tilde{p}, \vec{V} \rangle \equiv \phi$是标量场,在任意坐标系中:
\begin{equation}
    \phi = p_\alpha V^\alpha.
\label{equ5.59}
\end{equation}
根据求导的乘积法则可以算出$\nabla_\beta \phi$:
\begin{equation}
    \nabla_\beta \phi = \phi_{, \beta} = \frac{\partial p_{\alpha}}{\partial x^\beta} V^\alpha + p_\alpha \frac{\partial V^\alpha}{\partial x^\beta}.
\label{equ5.60}
\end{equation}
利用方程\eqref{equ5.50}可以将$\partial V^\alpha / \partial x^\beta$用$V\indices{^\alpha_{; \beta}}$(即$\nabla_\beta \vec{V}$的分量)替换:
\begin{equation}
    \nabla_\beta \phi = \frac{\partial p_\alpha}{\partial x^\beta} V^\alpha + p_{\alpha} V\indices{^\alpha_{; \beta}} - p_\alpha V^\mu \Gamma\indices{^\alpha_{\mu \beta}}.
\label{equ5.61}
\end{equation}
重新安排各项,并且对Christoffel符号项的傀儡指标重命名可得:
\begin{equation}
    \nabla_\beta \phi = \left( \frac{\partial p_\alpha}{\partial x^\beta} - p_\mu \Gamma\indices{^\mu_{\alpha \beta}} \right) V^\alpha + p_\alpha V\indices{^\alpha_{; \beta}}
\label{equ5.62}
\end{equation}
对于任意向量$\vec{V}$,上式中括号之外的两项\textbf{都已知是}张量的分量,因此,由于张量分量的乘法加法得到的结果都是新张量,因此括号中的项也是张量分量,这就是1形式$\tilde{p}$的协变导数:
\begin{shaded}
\begin{equation}
    (\nabla_\beta \tilde{p})_\alpha := (\nabla \tilde{p})_{\alpha \beta} := p_{\alpha; \beta} = p_{\alpha, \beta} - p_\mu \Gamma\indices{^\mu_{\alpha \beta}}.
\label{equ5.63}
\end{equation}
\end{shaded}
这样,\eqref{equ5.62}式写为
\begin{shaded}
\begin{equation*}
    \nabla_\beta (p_\alpha V^\alpha) = p_{\alpha; \beta} V^\alpha + p_\alpha V\indices{^\alpha_{; \beta}}
\end{equation*}
\end{shaded}
可见协变导数也服从乘积法则,和\eqref{equ5.60}式一样。实际上\textbf{必须如此},因为在直角坐标系中$\nabla$就是偏导数算符,上式在直角坐标系中退化为\eqref{equ5.60}式。

比较向量与1形式的协变导数\eqref{equ5.50}和\eqref{equ5.63}式:
\begin{align*}
    V\indices{^\alpha_{; \beta}} =& V\indices{^\alpha_{, \beta}} + V^\mu \Gamma\indices{^\alpha_{\mu \beta}}, \\
    p_{\alpha; \beta} =& p_{\alpha, \beta} - p_\mu \Gamma\indices{^\mu_{\alpha \beta}}.
\end{align*}
它们有相似也有不同。只要记住求导的分量$\beta$总是$\Gamma$的\textbf{最后一个}指标,则其它指标的上下位置不难确定。还需要记住符号的不同,一个辅助的办法是记住$\Gamma\indices{^\alpha_{\mu \beta}}$与基向量的导数有关(因此是正号),而1形式基的导数就对应$-\Gamma\indices{^\alpha_{\mu \beta}}$,符号的不同意味着1形式基的变换规律与基向量“相反”,从而使得缩并$\langle \tilde{\omega}^\alpha, \vec{e}_\beta \rangle = \delta\indices{^\alpha_\beta}$在坐标变换下\textbf{不变},其导数为零。

高阶张量的协变导数与方程\eqref{equ5.63}的推导过程同理,这里只给出结果:
\begin{shaded}
\begin{align}
    \nabla_\beta T_{\mu \nu} =& T_{\mu \nu, \beta} - T_{\alpha \nu} \Gamma\indices{^\alpha_{\mu \beta}} - T_{\mu \alpha} \Gamma\indices{^\alpha_{\nu \beta}} ; \label{equ5.64} \\
    \nabla_\beta A^{\mu \nu} =& A\indices{^{\mu \nu}_{, \beta}} + A^{\alpha \nu} \Gamma\indices{^\mu_{\alpha \beta}} + A^{\mu \alpha} \Gamma\indices{^\nu_{\alpha \beta}}; \label{equ5.65} \\
    \nabla_\beta B\indices{^\mu_\nu} =& B\indices{^\mu_{\nu, \beta}} + B\indices{^\alpha_\nu} \Gamma\indices{^\mu_{\alpha \beta}} - B\indices{^\mu_\alpha} \Gamma\indices{^\alpha_{\nu \beta}}. \label{equ5.66}
\end{align}
\end{shaded}
仔细观察会发现它们是\textbf{很有规律的}。在普通导数之外的每一项都分配一个$\Gamma$,上指标就像向量、下指标就像1形式那样进行处理。\eqref{equ5.64}式的几何意义是,$\nabla_\beta T_{\mu \nu}$是$\binom{0}{3}$张量$\nabla \bm{T}$的分量,其中$\bm{T}$是$\binom{0}{2}$张量。类似地,\eqref{equ5.65}式中,$\bm{A}$是$\binom{2}{0}$张量而$\nabla \bm{A}$是$\binom{2}{1}$张量,其分量为$\nabla_\beta A^{\mu \nu}$。

\section{Christoffel符号与度规}
\label{sec5.4}
上面的关于协变导数的推导没有涉及度规张量的任何内容,但是由于度规将向量与1形式联系起来,因此这两者的协变导数之间必然有一些关系。在直角坐标系中,1形式与相应的向量分量\textbf{相等},而$\nabla$在该系中就是普通求导算符,因此1形式与向量分量的协变导数在直角坐标系中必然相等。这意味着任一向量$\vec{V}$及其相应的1形式$\tilde{V} = \bm{g} (\vec{V}, \ )$在直角坐标系中有:
\begin{equation}
    \nabla_\beta \tilde{V} = \bm{g} (\nabla_\beta \vec{V}, \ ).
\label{equ5.67}
\end{equation}
上式是张量方程,在\textbf{所有}坐标系都成立,由此导出
\begin{equation}
    V_{\alpha; \beta} = g_{\alpha \mu} V\indices{^\mu_{; \beta}}
\label{equ5.68}
\end{equation}
上式就是\eqref{equ5.67}式的分量形式。

上面的论述也许不让你满意,下面一步一步重新推导一遍。用不带撇的指标$\alpha, \beta, \gamma, \dots$表示直角坐标系,带撇指标$\alpha', \beta', \gamma', \dots$表示\textbf{任意}坐标系。

我们的出发点是,在任意坐标系中,1形式与向量分量的对应关系为:
\begin{equation}
    V_{\alpha'} = g_{\alpha' \mu'} V^{\mu'}.
\label{equ5.69}
\end{equation}
而在直角坐标系中:
\begin{equation*}
    g_{\alpha \mu} = \delta_{\alpha \mu}, \quad V_{\alpha} = V^\alpha.
\end{equation*}
在直角坐标系中,Christoffel符号为零,因此在其中
\begin{equation*}
    V_{\alpha; \beta} = V_{\alpha, \beta}\quad V\indices{^\alpha_{; \beta}} = V\indices{^\alpha_{, \beta}}
\end{equation*}
由此可得
\[
    V_{\alpha; \beta} = V\indices{^\alpha_{; \beta}}
\]
这只在直角坐标系成立。为了将上式写成在任意坐标系都成立的形式,我们注意到在直角坐标系中
\begin{equation*}
    V\indices{^\alpha_{; \beta}} = g_{\alpha \mu} V\indices{^\mu_{; \beta}}
\end{equation*}
由上两式可得
\[
    V_{\alpha; \beta} = g_{\alpha \mu} V\indices{^\mu_{; \beta}}
\]
这就写成了张量方程,在所有坐标系成立,于是再次得到了\eqref{equ5.68}式:
\begin{equation}
    V_{\alpha'; \beta'} = g_{\alpha' \mu'} V\indices{^{\mu'}_{; \beta'}}
\label{equ5.70}
\end{equation}
上式有个重要推论。方程\eqref{equ5.69}对$\beta'$坐标求协变导数可得
\[
    V_{\alpha'; \beta'} = g_{\alpha' \mu'; \beta'} V^{\mu'} + g_{\alpha' \mu'} V\indices{^{\mu'}_{; \beta'}}    
\]
上式与\eqref{equ5.70}式比较(注意$\vec{V}$是任意向量)可得
\begin{shaded}
\begin{equation}
    g_{\alpha' \mu'; \beta'} \equiv 0,
\label{equ5.71}
\end{equation}
\end{shaded}
上式在任意坐标系都成立,它是方程\eqref{equ5.67}的推论,在直角坐标系中
\[
    g_{\alpha \mu; \beta} \equiv g_{\alpha \mu, \beta} = \delta_{\alpha \mu, \beta} = 0,
\]
它是个平凡的恒等式。而在一般坐标系中,度规的协变导数为零是不显然的,下面以极坐标系为例计算一下以验证结论是否正确。

利用\eqref{equ5.64}式可得(以下不带撇指标表示一般坐标系)
\begin{equation}
    g_{\alpha \beta; \mu} = g_{\alpha \beta, \mu} - \Gamma\indices{^\nu_{\alpha \mu}} g_{\nu \beta} - \Gamma\indices{^\nu_{\beta \mu}} g_{\alpha \nu}.
\label{equ5.72}
\end{equation}
在极坐标系中,考虑分量$\alpha = r, \beta = r, \mu = r$:
\[
    g_{rr; r} = g_{rr, r} - \Gamma\indices{^\nu_{rr}} g_{\nu r} - \Gamma\indices{^\nu_{rr}} g_{r \nu}.
\]
由于$g_{rr, r} = 0, \Gamma\indices{^\nu_{rr}} = 0, \forall\, \nu$,因此上式平淡地等于零。对于分量$\alpha = \theta, \beta = \theta, \mu = r$就不那么平淡了:
\[
    g_{\theta \theta; r} = g_{\theta \theta, r} - \Gamma\indices{^\nu_{\theta r}} g_{\nu \theta} - \Gamma\indices{^\nu_{\theta r}} g_{\theta \nu}.
\]
带入$g_{\theta \theta} = r^2, \Gamma\indices{^\theta_{\theta r}} = 1/r, \Gamma\indices{^r_{\theta r}} = 0$可得
\[
    g_{\theta \theta; r} = (r^2)_{, r} - \frac{1}{r} (r^2) - \frac{1}{r} (r^2) = 0.
\]
就像某种魔法让它等于零一样,但是这不是魔法,只是因为在直角坐标系中$g_{\alpha \beta, \mu} = 0$,而在任意坐标系中的$g_{\alpha \beta; \mu}$与$g_{\alpha \beta, \mu}$是\textbf{同一}张量$\nabla \bm{g}$在不同坐标系的分量。

现在应该总结一下我们干了些什么。利用欧几里得空间中平行的概念,我们引入了协变微分,然后证明了欧几里得空间的度规是协变不变的(\eqref{equ5.71}式)。之后讨论弯曲空间(Riemannian空间)的时候需要更仔细地研究平行,但是在那里\eqref{equ5.71}式\textbf{仍然有效},因此以该方程为基础的所有讨论内容也依然有效。

\subsection*{从度规计算Christoffel符号}
\eqref{equ5.72}式等于零有一个极其重要的推论,下面会看到,利用\eqref{equ5.72}式可以用$\Gamma\indices{^\mu_{\alpha \beta}}$表示$g_{\alpha \beta, \mu}$,而反过来也可以,即用$g_{\alpha \beta, \mu}$表示$\Gamma\indices{^\mu_{\alpha \beta}}$,这是计算Christoffel符号的简单方式。

为此,先得证明一个重要命题:\textbf{在任意坐标系中}$\Gamma\indices{^\mu_{\alpha \beta}} \equiv \Gamma\indices{^\mu_{\beta \alpha}}$。为了证明这一对称性,考虑任意张量场$\phi$,它的一阶导数$\nabla \phi$是分量为$\phi_{, \beta}$的1形式,它的二阶导数——$\binom{0}{2}$张量$\nabla \nabla \phi$的分量为$\phi_{, \beta; \alpha}$,在直角坐标系中,这些分量等于
\[
    \phi_{, \beta, \alpha} := \frac{\partial}{\partial x^\alpha} \frac{\partial}{\partial x^\beta} \phi,
\]
它的$\alpha, \beta$指标是对称的,因为偏导数互相对易。如果张量在一个坐标系中的指标对称,则在任意坐标系中的指标都对称。因此
\begin{equation}
    \phi_{, \beta; \alpha} = \phi_{, \alpha; \beta}
\label{equ5.73}
\end{equation}
在\textbf{任意}坐标系成立,根据定义式\eqref{equ5.63}将上式展开:
\[
    \phi_{, \beta, \alpha} - \phi_{, \mu} \Gamma\indices{^\mu_{\beta \alpha}} = \phi_{, \alpha, \beta} - \phi_{, \mu} \Gamma\indices{^\mu_{\alpha \beta}} \quad \text{在任意坐标系中。}
\]
而在\textbf{任意}坐标系中都有
\[
    \phi_{, \alpha, \beta} = \phi_{,\beta ,\alpha}
\]
由此可得
\[
    \Gamma\indices{^\mu_{\alpha \beta}} \phi_{, \mu} = \Gamma\indices{^\mu_{\beta \alpha}} \phi_{, \mu}
\]
对任意标量场$\phi$成立,这就证明了在任意坐标系中Christoffel符号的对称性:
\begin{shaded}
\begin{equation}
    \Gamma\indices{^\mu_{\alpha \beta}} = \Gamma\indices{^\mu_{\beta \alpha}}.
\label{equ5.74}
\end{equation}
\end{shaded}
利用上式与\eqref{equ5.72}式,进行一些指标上的旋转跳跃就能得到\eqref{equ5.72}式的逆形式,将方程\eqref{equ5.72}进行指标轮换:
\begin{align*}
    g_{\alpha \beta, \mu} =& \Gamma\indices{^\nu_{\alpha \mu}} g_{\nu \beta} + \Gamma\indices{^\nu_{\beta \mu}} g_{\alpha \nu}, \\
    g_{\alpha \mu, \beta} =& \Gamma\indices{^\nu_{\alpha \beta}} g_{\nu \mu} + \Gamma\indices{^\nu_{\mu \beta}} g_{\alpha \nu}, \\
    -g_{\beta \nu, \alpha} =& -\Gamma\indices{^\nu_{\beta \alpha}} g_{\nu \mu} - \Gamma\indices{^\nu_{\mu \alpha}} g_{\beta \nu}.
\end{align*}
以上三式相加,并利用$\bm{g}$的对称性$g_{\beta \nu} = g_{\nu \beta}$合并同类项:
\begin{align*}
    g_{\alpha \beta, \mu} +& g_{\alpha \mu, \beta} - g_{\beta \mu, \alpha} \\
    =& (\Gamma\indices{^\nu_{\alpha \mu}} - \Gamma\indices{^\nu_{\mu \alpha}}) g_{\nu \beta} + ( \Gamma\indices{^\nu_{\alpha \beta}} - \Gamma\indices{^\nu_{\beta \alpha}} ) g_{\nu \mu} + ( \Gamma\indices{^\nu_{\beta \mu}} + \Gamma\indices{^\nu_{\mu \beta}} ) g_{\alpha \nu}.
\end{align*}
根据$\Gamma$的对称性$\eqref{equ5.74}$式,上式右侧前两项为零,于是
\[
    g_{\alpha \beta, \mu} + g_{\alpha \mu, \beta} - g_{\beta \mu, \alpha} = 2 g_{\alpha \nu} \Gamma\indices{^\nu_{\beta \mu}}.
\]
就快要完成了。上式左右两边除以2,乘以$g^{\alpha \gamma}$(重复指标$\alpha$表示求和),并且利用
\[
    g^{\alpha \gamma} g_{\alpha \nu} \equiv \delta\indices{^\gamma_\nu}
\]
可得
\begin{shaded}
\begin{equation}
    \frac{1}{2} g^{\alpha \gamma} (g_{\alpha \beta, \mu} + g_{\alpha \mu, \beta} - g_{\beta \mu, \alpha}  ) = \Gamma\indices{^\gamma_{\beta \mu}}.
\label{equ5.75}
\end{equation}
\end{shaded}
这就是利用度规$\bm{g}$的分量及其导数表示Christoffel符号的表达式。例如,在极坐标系中:
\[
    \Gamma\indices{^\theta_{r \theta}} = \frac{1}{2} g^{\alpha \theta} (g_{\alpha r, \theta} + g_{\alpha \theta, r} - g_{r \theta, \alpha}).
\]
由于$g^{r \theta} = 0$以及$g^{\theta \theta} = r^{-2}$,可以导出
\begin{align*}
    \Gamma\indices{^\theta_{r \theta}} =& \frac{1}{2r^2} (g_{\theta r, \theta} + g_{\theta \theta, r} - g_{r \theta, \theta}) \\
    =& \frac{1}{2r^2} g_{\theta \theta, r} = \frac{1}{2r^2} (r^2)_{, r} = \frac{1}{r}.
\end{align*}
这与前面推导的结果相同。这种计算Christoffel符号的方法是坠吼的,方程\eqref{equ5.75}值得记住。尽管是在欧几里得空间推导的,但之后会看到它在弯曲空间中同样有效。


\subsection*{$\Gamma\indices{^\alpha_{\beta \mu}}$的张量性}
$\vec{e}_\alpha$是个向量,$\nabla \vec{e}_\alpha$是个$\binom{1}{1}$张量,它的分量是$\Gamma\indices{^\mu_{\alpha \beta}}$,这里的$\alpha$固定而$\mu, \beta$是分量指标:$\alpha$变化对应不同的张量$\nabla \vec{e}_\alpha$,而改变$\mu$或$\beta$只是对应同一张量的不同分量。

因此,可以把$\mu, \beta$视为分量指标,而把$\alpha$视为指明不同张量的标签,每个基向量$\vec{e}_\alpha$对应这样一个张量。然而这不太有用,因为在坐标变换之下,新旧坐标系的基向量不同,在新坐标系中重要的张量是\textbf{新的}$\nabla \vec{e}_{\beta'}$,它与旧的$\nabla \vec{e}_\alpha$的关系很复杂:它们是\textbf{不同的}张量,而非同一张量在不同坐标系的分量。因此一个坐标系中的$\Gamma\indices{^\mu_{\alpha \beta}}$不能由另一坐标系的$\Gamma\indices{^{\mu'}_{\alpha'
 \beta'}}$经过简单的变换得到。最简单的例子是在直角坐标系中所有的$\Gamma$都为零,而在一般的坐标系中非零。因此很多教材都说$\Gamma\indices{^\mu_{\alpha \beta}}$不是张量分量。从我们的观点来看,这话并不严格正确:$\Gamma\indices{^\mu_{\alpha \beta}}$是\textbf{一组}$\binom{1}{1}$\textbf{张量}$\nabla \vec{e}_\alpha$的$(\mu, \beta)$分量,只是不存在一个$\binom{1}{2}$张量的分量是$\Gamma\indices{^\mu_{\alpha \beta}}$,因此形如$\Gamma\indices{^\mu_{\alpha \beta}} V^\alpha $的表达式也不是单个张量的分量,而组合
 \[
    V\indices{^\beta_{, \alpha}} + V^\mu \Gamma\indices{^\beta_{\mu \alpha}}
\]
\textbf{是}张量$\nabla \vec{V}$的分量。



\section{非坐标基}
\label{sec5.5}
之前所有的讨论都默认了非直角坐标系的基向量是由直角坐标$(x, y)$变换为某个$(\xi, \eta)$产生的。然而,下面会看到,并非所有基向量场都可以通过这种方式得到,并且我们要研究需要什么条件(不多)才可以。本书几乎不适用非坐标基,但是在平坦空间曲线坐标系的标准教材中却经常使用它们,因此有必要对它们进行介绍。

\subsection*{极坐标基}
极坐标系的基向量的定义是
\[
    \vec{e}_{\alpha'} = \Lambda\indices{^\beta_{\alpha'}} \vec{e}_{\beta},
\]
其中带撇指标表示极坐标,不带撇的表示直角坐标。此外,
\[
    \Lambda\indices{^\beta_{\alpha'}} = \frac{\partial x^\beta}{\partial x^{\alpha'}},
\]
这里将直角坐标$\{ x^\beta \}$视为极坐标$\{  x^{\alpha'} \}$的函数。可以导出
\[
    \vec{e}_{\alpha'} \cdot \vec{e}_{\beta'} \equiv g_{\alpha' \beta'} \neq \delta_{\alpha' \beta'},
\]
这意味着极坐标系\textbf{并非}单位向量。

\subsection*{极坐标单位基}
便利起见,经常定义\textbf{单位}基向量。容易导出极坐标系的单位基向量为:
\begin{equation}
    \vec{e}_{\hat{r}} = \vec{e}_{r}, \quad \vec{e}_{\hat{\theta}} = \frac{1}{r} \vec{e}_\theta,
\label{equ5.76}
\end{equation}
相应的1形式基为
\begin{equation}
    \tilde{\omega}^{\hat{r}} = \trd r, \quad \tilde{\omega}^{\hat{\theta}} = r \trd \theta.
\label{equ5.77}
\end{equation}
米娜桑应该自行证明
\begin{equation}
\left.
\begin{split}
    \vec{e}_{\hat{\alpha}} \cdot \vec{e}_{\hat{\beta}} \equiv g_{\hat{\alpha} \hat{\beta}} = \delta_{\hat{\alpha} \hat{\beta}}, \\
    \tilde{\omega}^{\hat{\alpha}} \cdot \tilde{\omega}^{\hat{\beta}} \equiv g^{\hat{\alpha} \hat{\beta}} = \delta^{\hat{\alpha} \hat{\beta}}.
\end{split}
\right\}
\label{equ5.78}
\end{equation}
这样就定义了正交归一的向量与1形式基,用带帽号$\hat{\ }$的指标表示。现在问题来了,是否存在坐标系$(\xi, \eta)$使得
\begin{subequations}
\begin{alignat}{2}
    \vec{e}_{\hat{\alpha}} =& \vec{e}_\xi = \frac{\partial x}{\partial \xi} \vec{e}_x + \frac{\partial y}{\partial \xi} \vec{e}_y  && \label{equ5.79a} \\
    \vec{e}_{\hat{\theta}} =& \vec{e}_\eta = \frac{\partial x}{\partial \eta} \vec{e}_x + \frac{\partial y}{\partial \eta} \vec{e}_y &&\label{equ5.79b}
\end{alignat}
\end{subequations}
成立?

如果成立,则$\{ \vec{e}_{\hat{\alpha}}, \vec{e}_{\hat{\theta}} \}$就是坐标系$(\xi, \eta)$的基向量,可以称之为坐标基;如果证明满足条件的$(\xi, \eta)$不存在,则这些向量是非坐标基。用相应的1形式基证明会更加容易,因此,下面要尝试找到坐标系$(\xi, \eta)$使得
\begin{equation}
\left.
\begin{split}
    \tilde{\omega}^{\hat{r}} =& \trd \xi = \frac{\partial \xi}{\partial x} \trd x + \frac{\partial \xi}{\partial y} \trd y, \\
    \tilde{\omega}^{\hat{\theta}} =& \trd \eta = \frac{\partial \eta}{\partial x} \trd x + \frac{\partial \eta}{\partial y} \trd y.
\end{split}
\right\}
\label{equ5.80}
\end{equation}
$\tilde{\omega}^{\hat{r}}, \tilde{\omega}^{\hat{\theta}}$关于$\trd r, \trd \theta$的关系已知,因此根据\eqref{equ5.26}和\eqref{equ5.27}式可得
\begin{equation}
\left.
\begin{split}
    \tilde{\omega}^{\hat{r}} =& \trd r = \cos \theta \trd x + \sin \theta \trd y, \\
    \tilde{\omega}^{\hat{\theta}} =& r \trd \theta = -\sin \theta \trd x + \cos \theta \trd y.
\end{split}
\right\}
\label{equ5.81}
\end{equation}
(不难验证$\tilde{\omega}^{\hat{r}}, \tilde{\omega}^{\hat{\theta}}$是正交的)

如果$(\xi, \theta)$存在,则
\begin{equation}
    \frac{\partial \eta}{\partial x} = -\sin \theta, \quad \frac{\partial \eta}{\partial y} = \cos \theta.
\label{equ5.82}
\end{equation}
如果上式成立,则混合导数相等:
\begin{equation}
    \frac{\partial}{\partial y} \frac{\partial \eta}{\partial x} = \frac{\partial}{\partial x} \frac{\partial \eta}{\partial y}.
\label{equ5.83}
\end{equation}
这意味着
\begin{equation}
    \frac{\partial}{\partial y} (-\sin \theta) = \frac{\partial}{\partial x} (\cos \theta)
\label{equ5.84}
\end{equation}
也就是
\begin{equation*}
    \frac{\partial}{\partial y} \left( \frac{y}{\sqrt{x^2 + y^2}} \right) + \frac{\partial}{\partial x} \left( \frac{x}{\sqrt{x^2 + y^2}} \right)  = 0.
\end{equation*}
上式当然\textbf{不可能成立}。因此$\xi, \eta$\textbf{不存在}:无法找到符合条件的坐标系。(If this manner of proof is surprising, try it on $\trd r$ and $\trd \theta$ themselves.)

关于曲线坐标系中向量微积分的教材几乎都采用正交归一基,而非坐标基。例如,某个向量在极坐标系\textbf{坐标基}(coordinate basis, PC)下的分量为
\begin{equation}
    \vec{V} \xrightarrow[\text{PC}]{ } (a, b) = \{ V^{\alpha'} \},
\label{equ5.85}
\end{equation}
则它在\textbf{正交基} (orthonormal basis, PO)下的分量为:
\begin{equation}
    \Vec{V} \xrightarrow[\text{PO}]{ } (a, rb) = \{ V^{\hat{\alpha}} \}.
\label{equ5.86}
\end{equation}
例如,相关教材计算向量散度的公式为:(与我们导出的\eqref{equ5.56}式有所不同)
\begin{equation}
    \nabla \cdot \bm{V} = \frac{1}{r} \frac{\partial}{\partial r} (r V^{\hat{r}}) + \frac{1}{r} \frac{\partial}{\partial \theta} V^{\hat{\theta}}.
\label{equ5.87}
\end{equation}
\eqref{equ5.56}和\eqref{equ5.87}式的不同只是因为$\Vec{V}$的基向量不同。

\subsection*{非坐标基总论}

\subsection*{本书的非坐标基}


\section{展望下一步}
\label{sec5.6}

\section{扩展阅读}
\label{sec5.7}

\section{习题}
\label{sec5.8}