\chapter{弯曲时空中的物理学}
\label{chap7}

\section{微分几何与引力理论的对应}
\label{sec7.1}
将物理理论用数学形式表达,本质上是将数学概念与可观测物理量联系起来。因此我们首先要把之前建立的几何概念与现实世界的引力进行对应,这之前已经多少讨论过了,例如我们假设时空是个微分流形,并且已经证明了在非均匀引力场下不存在全局惯性系,在这些论述之后是如下的两条对应:

\begin{shaded}
\begin{enumerate}
    \item[(\uppercase\expandafter{\romannumeral1})] 时空(所有事件的集合)是带有度规的四维流形。
    \item[(\uppercase\expandafter{\romannumeral2})] 度规可以由杆与钟测量。用杆测量的相邻两事件点(它们的时间相同)的空间间距是$|\rd \vec{x} \cdot \rd \vec{x}|^{1/2}$,用钟测量的时间相近的两事件点(它们的空间位置相同)的时间间隔是$|-\rd \vec{x} \cdot \rd \vec{x}|^{1/2}$。
\end{enumerate}
\end{shaded}
一般不存在一个坐标系使得$\rd \vec{x} \cdot \rd \vec{x} = -(\rd x^0)^2 + (\rd x^1)^2 + (\rd x^2)^2 + (\rd x^3)^2$处处成立。另一方面,我们论证了这种坐标系在\textbf{局域上}是存在的。这显然对应着弯曲流形,在其中任一点附近都存在坐标系局域的与Minkowski时空相似(例如向量的点乘形式)。

据此可以进行如下的对应条件:
\begin{shaded}
\begin{enumerate}
    \item[(\uppercase\expandafter{\romannumeral3})] 在任一事件点附近,存在合适的坐标系使得时空的度规为Lorentz形式$\eta_{\alpha \beta}$。
\end{enumerate}
\end{shaded}
在钦定了这种表示时空的方式之后,还需要完成两点才能得到完整的理论。第一,必须说明物理对象(粒子、电场、流体等等)在弯曲时空中如何表现;第二,必须说明时空中的物体是怎样产生或着决定曲率的。

作为物理理论的例子,我们先来考虑牛顿引力理论。牛顿理论的时空是三维欧几里得空间加上一维时间轴(数学上的记号是$R^3 \times R$)。不存在整体时空流形上的度规,空间距离来自欧几里得空间自带的普通度规、时间由绝对的“宇宙钟”测量。速度不同的观测者地位相同:这是伽利略力学里的相对性。因此不存在绝对静止的参照物,并且不同观者对于两个不同时的事件是否发生于同一地点的结论可能不同。但是所有观者的同时性——即两个事件是否在同一时间(时间轴的同一切片)发生——是绝对的,所以两个事件的“时间间隔”意味着包含两个事件的欧几里得切面之间经过的时间。它与事件的空间位置无关,因此牛顿引力理论中的时间是绝对的:所有观者,无论位置或运动状态如何,观测到的两事件的时间间隔都相同。类似地,两事件的“空间间隔”的含义是两事件在欧几里得空间的距离。如果两个时间是同时的(发生于时间轴的同一欧几里得切片),则利用那个切片的空间度规即可计算空间间隔,所有观测者的结果相同。如果两个事件发生的时间不同,则每个观者取他所观测到的事件的切片并计算空间间隔,不同观者观测到的事件的坐标不同,但距离都相同。

然而,在牛顿理论中无法将时间与距离观测结合起来: there was no invariant measure of the length of a general curve that changed position and time as it went along. 没有将时间与空间距离相互转换的不变量,也就无法进行结合。爱因斯坦为相对论引入了不变量——光速,它将空间与时间的观测量统一起来。爱因斯坦四维时空的结构实际上比牛顿的绝对时空更加简单!

在绝对时空的模型下,牛顿给出了描述物体受引力影响的定律:$\bm{F} = -m \bm{a}$,其中$\bm{F} = -m \nabla \phi$是物体在给定引力场$\phi$当中所受的引力。牛顿也给出了物体是如何产生引力场$\phi$的:$\nabla^2 \phi = 4\pi G \rho$。时空的相对论描述必须给出这两个定律的对应版本,第二个式子的对应在下一章讨论,本章研究一个给定的度规是如何影响时空中的物体的。

之前讨论过粒子运动的简单情况。我们知道粒子在引力场中的“加速度”与粒子质量无关,因此在附近的自由下落参考系中粒子加速度为零,这就对应于局域惯性系。因为自由下落粒子在该系的加速度为零,因此粒子(至少是局域地)沿直线运动。局域惯性系的直线正是弯曲流形的测地线的\textbf{定义}。这样我们得到了关于粒子受度规影响的第一条假设:\footnote{WEP更普遍的定义没有涉及弯曲时空,而只是说粒子在引力场的下落速度相同(不依赖于粒子质量与位置)。但是爱因斯坦等效原理(假设\uppercase\expandafter{\romannumeral4}$'$意味着引力可以由时空曲率表示,因此我们以弯曲时空作为讨论的出发点。}

\begin{shaded}
\begin{enumerate}
    \item[(\uppercase\expandafter{\romannumeral4})] \textbf{弱等效原理 (Weak Equivalence Principle, WEP)}:自由下落粒子沿时空的类时测地线运动。
\end{enumerate}
\end{shaded}
“自由下落”意味着粒子不受其它力影响,例如电场力等等。其它所有的力与引力的区别在于,\textbf{存在着}不受其它力作用的粒子。因此弱等效原理(假设\uppercase\expandafter{\romannumeral4})是很强的陈述,它可以被实验检验。WEP饱经、并且一直在经受高精度的检验,其中典型的实验例如比较由不同材料组成的物体的下落速度;当前的对加速度差别的实验限制已经到了$10^{13}$量级(Will 2006)。因此WEP是被实验验证的精度最高的物理定律之一。人们计划利用卫星搭载的实验来将精度提高到$10^{-18}$量级。

不过,WEP只针对粒子,其它对象(例如流体)是如何受非平直度规影响的?为此需要将\uppercase\expandafter{\romannumeral4}进行推广:

\begin{shaded}
\begin{enumerate}
    \item[(\uppercase\expandafter{\romannumeral4})$'$] \textbf{爱因斯坦等效原理 (Einstein Equivalence Principle, EEP)}:任何不涉及引力的、局域的物理实验,在(弯曲时空中的)自由下落惯性系的结果与在平直时空(狭义相对论有效)中的结果相同。
\end{enumerate}
\end{shaded}
其中,“局域”的含义是实验不涉及场,例如电场,场延伸到一大片时空区域,超出了局域惯性系的有效范围。所有的\textbf{局域}物理结果在自由下落惯性系的结果,与狭义相对论的结果相同。引力\textbf{局域上}没有引入任何新内容,所有引力的效应要在时空中延伸一定的区域才能感受到,这也被实验严格检验过了(Will 2006)。

(未完成)

This may seem strange to someone used to blaming gravity for making it hard to climb
stairs or mountains, or even to get out of bed! But these local effects of gravity are, in
Einstein’s point of view, really the effects of our being pushed around by the Earth and
objects on it. Our ‘weight’ is caused by the solid Earth exerting forces on us that prevent us
from falling freely on a geodesic (weightlessly, through the floor). This is a very reasonable
point of view. Consider astronauts orbiting the Earth. At an altitude of some 300 km, they
are hardly any further from the center of the Earth than we are, so the strength of the
Newtonian gravitational force on them is almost the same as on us. But they are weightless,
as long as their orbit prevents them encountering the solid Earth. Once we acknowledge
that spacetime has natural curves, the geodesics, and that when we fall on them we are in
freefallandfeelnogravity,thenwecandisposeoftheNewtonianconceptofagravitational
force altogether. We are only following the natural spacetime curve.


The true measure of gravity on the Earth are its tides. These are nonlocal effects, because
they arise from the difference of the Moon’s Newtonian gravitational acceleration across
the Earth, or in other words from the geodesic deviation near the Earth. If the Earth were
permanently cloudy, an Earthling would not know about the Moon from its overall grav-
itational acceleration, since the Earth falls freely: we don’t feel the Moon locally. But
Earthlings could in principle discover the Moon even without seeing it, by observing and
understanding the tides. Tidal forces are the only measurable aspect of gravity.

爱因斯坦等效原理数学上的含义是(粗略地说),如果一条局域的物理定律在SR中写成了张量方程,则该定律在弯曲时空的局域惯性系中的数学形式也是如此。

这一原理经常被称作“逗号变分号规则”,因为如果定律的SR形式含有导数(逗号),则它在局域惯性系中也是如此。为了让方程在\textbf{任意}坐标系都有效,只需要在局域惯性系中把普通导数换成协变导数(分号),\footnote{当然,这样做的理由是Christoffel符号在局域惯性系的原点处等于零,前面用过很多次了。——译者} 这是一种超级简单地推广物理定律的方式。特别地,这一原理禁止了“曲率耦合”:例如热力学在弯曲时空的真正形式可能含有Riemann张量,而在SR中Riemann张量的相关内容会消失不见。假设(\uppercase\expandafter{\romannumeral4})$'$不允许Riemann张量有关的项出现在方程中。

下面举例说明怎样将((\uppercase\expandafter{\romannumeral4})$'$)转换为数学形式,以流体动力学(它是本课程的主要研究对象)为例。SR中的粒子数守恒定律表示为
\begin{equation}
    (n U^\alpha)_{, \alpha} = 0,
\label{equ7.1}
\end{equation}
其中$n$是瞬时共动参考系(MCRF)的粒子数密度,而$U^\alpha$是流体元的四速。在弯曲时空中的任意事件处,可以可以找到与该事件处的流体元瞬时共动的局域惯性系,并且按类似的方式定义$n$,同样,可以像SR那样定义四速$\vec{U}$是那个坐标系的时间基向量。于是,根据爱因斯坦等效原理,粒子数守恒律在局域惯性系的形式\textbf{正是}\eqref{equ7.1}式。由于Christoffel符号在给定事件点(该点是局域惯性系的原点)处等于零,\eqref{equ7.1}式等价于
\begin{equation}
    (n U^\alpha)_{; \alpha} = 0.
\label{equ7.2}
\end{equation}
这种张量方程的形式在\textbf{任何}坐标系都成立,从而利用上式就可以在任意坐标系中计算粒子数守恒,还要记住上式源于爱因斯坦等效原理。

这样,我们将粒子数守恒律推广到了弯曲时空。在之后需要的场合,我们总是利用这种方法推广物理定律。

上述内容仅仅是张量的游戏,还是蕴含着新的物理意义?弯曲时空中的粒子守恒律可能是\eqref{equ7.2}以外的形式吗?有可能。考虑如下形式:
\begin{equation}
    (n U^\alpha)_{; \alpha} = qR^2,
\label{equ7.3}
\end{equation}
其中$R$是Ricci标量,定义在\eqref{equ6.92}式,它是Riemann张量的两次缩并;$q$是常数。由于平直时空的Riemann张量为零,因此上式也会退化到SR形式\eqref{equ7.1}式。但是在弯曲时空中,上式预言的内容完全不同:曲率可以产生或湮灭粒子(与$q$的符号有关)。上两式都和SR的形式相符。爱因斯坦等效原理断言,应该把\eqref{equ7.1}式推广为尽量简单的形式,也就是\eqref{equ7.2}。当然,决定\eqref{equ7.2}和\eqref{equ7.3}式哪个正确要靠实验或者天文观测。本书采用绝大多数的观点——爱因斯坦等效原理是正确的。没有什么观测性证据反对它。

类似地,SR中的熵守恒定律是
\begin{equation}
    U^\alpha S_{, \alpha} = 0.
\label{equ7.4}
\end{equation}
因为标量$S$的协变导数不含Christoffel符号,所以上式在弯曲时空的形式\textbf{不变}。

最后,四动量守恒定律为
\begin{equation}
    T\indices{^{\mu \nu}_{, \nu}} = 0.
\label{equ7.5}
\end{equation}
推广到弯曲时空的形式为
\begin{equation}
    T\indices{^{\mu \nu}_{; \nu}} = 0.
\label{equ7.6}
\end{equation}
其中能动张量的定义与之前相似:
\begin{equation}
    T^{\mu \nu} = (\rho + p) U^\mu U^\nu + p g^{\mu \nu}.
\label{equ7.7}
\end{equation}
(注意,$g^{\mu \nu}$在局域惯性系等于平直时空度规$\eta^{\mu \nu}$.)

\section{轻微弯曲时空的物理学}
\label{sec7.2}
为了研究爱因斯坦等效原理((\uppercase\expandafter{\romannumeral4})$'$)对粒子或流体运动的影响,我们必须知道时空流形的度规。由于还没讲度规是如何产生的,因此目前必须钦点一个在之后推导的度规作为出发点。之后会看到,\textbf{弱引力场}(用牛顿理论的话来说,就是粒子的引力势能远小于粒子的静质能)的度规完全由牛顿引力势$\phi$所确定,其形式为
\begin{shaded}
\begin{equation}
    \rd s^2 = -(1 + 2\phi)^2 \rd t^2 + (1 - 2\phi) (\rd x^2 + \rd y^2 + \rd z^2).
\label{equ7.8}
\end{equation}
\end{shaded}
($\phi$的值小于零,从而在距离质点$M$很远的地方有$\phi = -GM / r$.) 注意弱引力场的条件意味着$|m \phi| \ll m$,即$|\phi| \ll 1$。度规\eqref{equ7.8}式实际上只是精确到$\phi$的一阶量,因此之后的推导也只保留到$\phi$的一阶量,而略去二阶及以上小量。

下面计算自由下落粒子的运动情况。粒子的四动量记作$\vec{p}$,它等于$m \vec{U}$(除了无质量粒子),其中$\vec{U} = \rd \vec{x} / \rd \tau$。根据弱等效原理((\uppercase\expandafter{\romannumeral4})),自由粒子的轨迹是测地线,我们知道粒子的固有时是测地线的一种仿射参量。因此$\vec{U}$必须满足测地线方程:
\begin{equation}
    \nabla_{\vec{U}} \vec{U} = 0.
\label{equ7.9}
\end{equation}
便利起见,注意到固有时的常数倍$\tau / m$也是测地线的仿射参量,于是$\rd \vec{x} / \rd (\tau/m)$也是满足测地线方程的向量,这个向量就是$m \rd \vec{x} / \rd \tau = \vec{p}$。因此自由粒子的运动方程可以写为
\begin{equation}
    \nabla_{\vec{p}} \vec{p} = 0.
\label{equ7.10}
\end{equation}
这个方程也能用于光子,尽管光子的$\vec{p}$定义良好,但是由于$m = 0$所以$\vec{U}$没有定义。

设粒子在\eqref{equ7.8}式的坐标系中的速度是非相对论性的,下面来计算\eqref{equ7.10}式的非相对论近似。首先考虑方程的0分量,注意沿$\vec{p}$的普通导数等于$m$乘以沿$\vec{U}$的普通导数,即$m \rd / \rd \tau$:
\begin{equation}
    m \frac{\rd}{\rd \tau} p^0 + \Gamma\indices{^0_{\alpha \beta}} p^\alpha p^\beta = 0.
\label{equ7.11}
\end{equation}
粒子的速度是非相对论性的,即$p^0 \gg p^1$,于是\eqref{equ7.11}式近似为
\begin{equation}
    m \frac{\rd}{\rd \tau} p^0 + \Gamma\indices{^0_{00}} (p^0)^2 = 0.
\label{equ7.12}
\end{equation}
下面计算$\Gamma\indices{^0_{00}}$:
\begin{equation}
    \Gamma\indices{^0_{00}} = \frac{1}{2} g^{0 \alpha} (g_{\alpha 0, 0} + g_{\alpha 0, 0} - g_{00, \alpha}).
\label{equ7.13}
\end{equation}
由于矩阵$[g_{\alpha \beta}]$是对角的,因此它的逆矩阵$[g^{\alpha \beta}]$也是对角的,逆矩阵元等于相应对角元的倒数。因此$g^{0 \alpha}$的非零元只有$\alpha = 0$,于是\eqref{equ7.13}式化为
\begin{align}
    \Gamma\indices{^0_{00}} &= \frac{1}{2} g^{00} g_{00, 0} = \frac{1}{2} \frac{1}{-(1 + 2\phi)} (-2\phi)_{, 0} \notag \\
    &= \phi_{, 0} + 0(\phi^2). \label{equ7.14}
\end{align}
可见$\Gamma\indices{^0_{00}}$为$\phi$的一阶量,因此\eqref{equ7.12}式中与$\Gamma\indices{^0_{00}}$相乘的$(p^0)^2$只需保留$\phi$的零阶量——$m^2$,这样\eqref{equ7.12}式化为
\begin{equation}
    \frac{\rd}{\rd \tau} p^0 = -m \frac{\partial \phi}{\partial \tau}.
\label{equ7.15}
\end{equation}
由于$p^0$即为粒子在该坐标系的能量,因此上式意味着如果引力场不随时间变化,则自由粒子的能量守恒。这个结论在牛顿理论中也成立。然而,必须注意$p^0$只是粒子相对于该坐标系的能量。

测地线的空间分量对应于牛顿理论的$\bm{F} = m \bm{a}$:
\begin{equation}
    p^\alpha p\indices{^i_{, \alpha}} + \Gamma\indices{^i_{\alpha \beta}} p^\alpha p^\beta = 0,
\label{equ7.16}
\end{equation}
在$\Gamma$的求和中利用非相对论性近似$p^0 \gg p^1$将上式化为
\begin{equation}
    m \frac{\rd p^i}{\rd \tau} + \Gamma\indices{^i_{00}} (p^0)^2 = 0.
\label{equ7.17}
\end{equation}
仍然利用近似$(p^0)^2 = m^2$带入得到\ \textit{画外音:其实应该先算Christoffle符号$\Gamma\indices{^i_{00}}$确定它是一阶小量,才敢把与它相乘的$(p^0)^2$只保留零阶项 }
\begin{equation}
    \frac{\rd p^i}{\rd \tau} = -m \Gamma\indices{^i_{00}}.
\label{equ7.18}
\end{equation}
下面计算Christoffel符号:
\begin{equation}
    \Gamma\indices{^i_{00}} = \frac{1}{2} g^{i \alpha} (g_{\alpha 0, 0} + g_{\alpha 0, 0} - g_{00, \alpha}).
\label{equ7.19}
\end{equation}
由于$[g^{\alpha \beta}]$是对角矩阵,因此它的逆矩阵为
\begin{equation}
    g^{i \alpha} = (1 - 2\phi)^{-1} \delta^{i \alpha},
\label{equ7.20}
\end{equation}
由此可得
\begin{equation}
    \Gamma\indices{^i_{00}} = \frac{1}{2} (1 - 2\phi)^{-1} \delta^{ij} (2g_{j0, 0} - g_{00, j}),
\label{equ7.21}
\end{equation}
其中已经把$\alpha$换成了$j$,因为$\delta^{i0}$等于零。注意到$g_{j0} \equiv 0$,由此可得
\begin{align}
    \Gamma\indices{^i_{00}} &= -\frac{1}{2} g_{00, j} \delta^{ij} + 0(\phi^2) \label{equ7.22} \\
    &= -\frac{1}{2} (-2\phi)_{, j} \delta^{ij} \label{equ7.23}
\end{align}
这样,运动方程\eqref{equ7.17}化为
\begin{shaded}
\begin{equation}
    \frac{\rd p^i}{\rd \tau} = -m \phi_{, j} \delta^{ij}.
\label{equ7.24}
\end{equation}
\end{shaded}
牛顿理论的引力为$-m \nabla \phi$,可见上式与牛顿理论一致。因此在高阶效应很小观测不到的情况下,广义相对论对行星轨道的预言仍然是开普勒观测的那样,后面会看到对大部分行星都是如此,除了水星,它的高阶效应可以观测到。

上面的能量守恒方程与运动方程都源于两个近似:度规近似为Minkowski度规($|\phi| \ll 1$),粒子的速度是非相对论性的($p^0 \gg p^i$)。这两个限制正是牛顿理论的适用范围,这就确保了我们重新导出牛顿方程。然而这不是有什么魔法,\textbf{必须回到}牛顿方程,因为我们知道自由粒子在自由下落参考系中沿直线运动。

对于其它系统,可以进行类似的计算以证明在合适的近似下回到牛顿理论。例如,本章习题5是关于理想流体的例子,注意流体的非相对论性不仅意味着流体的运动速度很小,而且流体粒子的随机运动(热运动)速度也是非相对论的,这意味着$p \ll \rho$。

相对论与旧的牛顿理论的这种在合适极限下的对应关系相当重要。\textbf{任何}新理论必须在旧理论适用的范围内与旧理论一致。等效原理加上\eqref{equ7.8}式的度规做到了这一点。

\section{Curved intuition}
\label{sec7.3}
尽管在合适的极限下,弯曲时空与牛顿理论的预言结果相同,但是它们在概念上差别很大,必须逐渐理解新理论的观点。

第一个差别在于是否有优越的坐标系。牛顿理论\textbf{以及}SR中,惯性系是优越的。由于“速度”不能被局域测量但是“加速度”可以,这两个理论选出了一组特殊的时空坐标系,粒子在其中既没有物理加速度(即$\rd \vec{U} / \rd \tau = 0$)也没有坐标加速度($\rd^2 x^i / \rd t^2 = 0$)。在新理论中,全局惯性系(即任意$\rd \vec{U} / \rd \tau = 0$的粒子都满足$\rd^2 x^i / \rd t^2 = 0$的坐标系)不存在,因此所有坐标系平权。利用Christoffel符号可以从$\rd^2 x^i / \rd t^2$这样坐标依赖的量得到$\rd \vec{U} / \rd \tau$这样不依赖坐标系的量。因此我们不需要也\textbf{不应该}建立坐标依赖的思维方式。

第二个差别在于能量和动量。在牛顿理论、SR与新的几何引力理论中,每个粒子都有确定的能量与动量,而它们的值依赖于在哪个坐标系计算。在后两个理论中,能量动量是同一个四维向量$\vec{p}$的分量。在SR中,系统总的四维动量是系统所有粒子的四动量之和$\sum_i \vec{p}_{(i)}$。但是在弯曲时空中,定义在不同点的向量\textbf{不能}简单地相加,因为不知道怎么加:两个向量只有在同一点进行比较才能判断是否平行,而向量从一点平移到另一点的值取决于沿哪条曲线平移,因此\textbf{不存在}将所有$\vec{p}$相加的不变的方式,即使一个系统有确定的四动量,事情也不会像SR那样容易。

可以发现,在空间上有界的系统(即孤立系统)的总能量与总动量\textbf{可以}定义,其定义方式之后讨论。系统的\textbf{总}质能不等于每个粒子的能量之和,这可以通过牛顿理论的情况看出:各粒子之间存在着自引力势能(引力自能),它是负的量,等于将组成系统的粒子从彼此相距无穷远移动到组成系统的过程中所得到的功。要将这种能量考虑在内,就不能只计算单个粒子,还要考虑时空几何。然而,引力势能的概念本身没有在新理论的图样中良好定义:它某种意味上表示了粒子能量之和与系统总质能的差值,但是由于粒子的能量之和没有良好定义,引力势能也没有。一般来说,只有系统\textbf{总的}能量-动量和单个粒子的四动量是可定义的。


\section{守恒量}
\label{sec7.4}
前面关于能量的讨论让人想要研究一个粒子或系统的守恒量。对于粒子,必须意识到引力在旧观点中是一个“力”,因此粒子在引力作用下动能、势能不守恒。相应的,在新观点下,不会有哪个坐标系使得$\vec{p}$在其中的分量沿着粒子轨迹不变。存在一种值得注意的例外,它十分重要,值得详细讨论。

将测地线方程的$\vec{p}$的上指标“降低”:
\begin{equation}
    p^\alpha p_{\beta; \alpha} = 0,
\label{equ7.25}
\end{equation}
将协变导数展开:
\begin{equation*}
    p^\alpha p_{\beta, \alpha} - \Gamma\indices{^\gamma_{\beta \alpha}} p^\alpha p_\gamma = 0,
\end{equation*}
移项得
\begin{equation}
    m \frac{\rd p_\beta}{\rd \tau} = \Gamma\indices{^\gamma_{\beta \alpha}} p^\alpha p_\gamma.
\label{equ7.26}
\end{equation}
等号右侧的项可以简化:
\begin{align}
    \Gamma\indices{^\gamma_{\alpha \beta}} p^\alpha p_\gamma &= \frac{1}{2} g^{\gamma \nu} (g_{\nu \beta, \alpha} + g_{\nu \alpha, \beta} - g_{\alpha \beta, \nu}) p^\alpha p_\gamma \notag \\
    &= \frac{1}{2} (g_{\nu \beta, \alpha} + g_{\nu \alpha, \beta} - g_{\alpha \beta, \nu}) g^{\gamma \nu} p_\gamma p^\alpha \notag \\
    &= \frac{1}{2} (g_{\nu \beta, \alpha} + g_{\nu \alpha, \beta} - g_{\alpha \beta, \nu}) p^\nu p^\alpha. \label{equ7.27}
\end{align}
$p^\nu p^\alpha$的$\nu \alpha$指标对称,而上式括号中的第一、三项$(g_{\nu \beta, \alpha} - g_{\alpha \beta, \nu})$的$\nu \alpha$指标反称,对称张量与反称张量缩并结果为零,这样只余下中间一项:
\begin{equation}
    \Gamma\indices{^\gamma_{\beta \alpha}} p^\alpha p_\gamma = \frac{1}{2} g_{\nu \alpha, \beta} p^\nu p^\alpha.
\label{equ7.28}
\end{equation}
因此,不失一般性,测地线方程化为
\begin{shaded}
\begin{equation}
    m \frac{\rd p_\beta}{\rd \tau} = \frac{1}{2} g_{\nu \alpha, \beta} p^\nu p^\alpha.
\label{equ7.29}
\end{equation}
\end{shaded}
由此可得如下重要结论:\textbf{如果$g_{\alpha \nu}$的所有分量与坐标$x^\beta$($\beta$是给定的值)无关,则$p_\beta$沿着任何自由粒子轨迹是常量。}

例如,对于静态(不依赖于时间)引力场,可以找到坐标系使得其中的度规分量不依赖时间,因此系统的$p_0$守恒。于是$p_0$(实际上是$-p_0$)通常称为粒子的“能量”,\textbf{而不需要}指出是“在该坐标系中的”能量。注意,静态度规在某个坐标系的分量可以依赖时间,例如从“好的”坐标系进行与时间有关的坐标变换就行了。实际上,大部分自由下落局域惯性系都是这种系,因为自由下落粒子受到随位置变化的引力场的影响,因此该系中的度规随时间变化。度规分量为静态的坐标系是特别的,它通常是地球上的“实验室坐标系”。因此这个系中的$p_0$与通常在实验室观测到的粒子能量有关,它包含了粒子的引力势能,下面就来进行论证。考虑
\begin{align}
    \vec{p} \cdot \vec{p} &= -m^2 = g_{\alpha \beta} p^\alpha p^\beta \notag \\
    &= -(1 + 2\phi) (p^0)^2 + (1 - 2\phi) \big[ (p^x)^2 + (p^y)^2 + (p^z)^2 \big], \label{equ7.30}
\end{align}
上式已经利用了度规$\eqref{equ7.8}$式,由上式可得
\begin{equation}
    (p^0)^2 = \big[ m^2 + (1 - 2\phi) (\bm{p}^2) \big] (1 + 2\phi)^{-1},
\label{equ7.31}
\end{equation}
其中将$(p^x)^2 + (p^y)^2 + (p^z)^2$简记为$\bm{p}^2$。利用近似$|\phi| \ll 1, |\bm{p}| \ll m$,上式简化为
\begin{align}
    (p^0)^2 & \approx m^2 \left(1 - 2\phi + \frac{\bm{p}^2}{m^2} \right) \notag \\
\intertext{亦即}
    p^0 & \approx m \left( 1 - \phi + \frac{\bm{p}^2}{2m^2} \right). \label{equ7.32}
\end{align}
将指标降低:
\begin{equation}
    p_0 = g_{0 \alpha} p^\alpha = g_{00} p^0 = -(1 + 2\phi) p^0,
\label{equ7.33}
\end{equation}
\begin{shaded}
\begin{equation}
    -p_0 \approx m \left( 1 + \phi + \frac{\bm{p}^2}{2m^2} \right) = m + m\phi + \frac{\bm{p}^2}{2m}.
\label{equ7.34}
\end{equation}
\end{shaded}
右侧第一项是粒子的静质能,第二、三项分别是牛顿形式的引力势能和动能。可见,$p_0$沿粒子轨迹守恒是牛顿理论中能量守恒的推广。

注意,对于\textbf{一般的}引力场,不存在坐标系使得其中的度规分量是静态的,\footnote{这种坐标系的不存在性容易证明。$4 \times 4$对称度规分量矩阵有10个独立分量,而坐标变换(四个函数$x^{\bar{\alpha}} (x^\mu)$)只引入了改变度规分量的四个自由度。只有特殊的度规才能通过坐标变换将所有分量变为不依赖于时间。 }因此不能定义守恒的能量。

类似地,如果度规具有轴对称性,即存在坐标系使得其中的度规分量$g_{\alpha \beta}$不依赖于旋转角$\psi$,则$p_\psi$是守恒的,它是粒子的角动量。在非相对论极限下
\begin{equation}
    p_\psi = g_{\psi \psi} p^\psi \approx g_{\psi \psi} m \frac{\rd \psi}{\rd t} \approx mg_{\psi \psi} \Omega,
\label{equ7.35}
\end{equation}
其中$\Omega$是粒子的角速度。对于接近平直的度规,在柱坐标系$(r, \psi, z)$中:
\begin{equation}
    g_{\psi \psi} = \vec{e}_\psi \cdot \vec{e}_\psi \approx r^2,
\label{equ7.36}
\end{equation}
因此相应的守恒量
\begin{equation}
    p_\psi \approx mr^2 \Omega.
\label{equ7.37}
\end{equation}
这就是牛顿理论的角动量。

关于粒子守恒律的讨论告一段落。流体的推导过程类似,因为流体就是大量粒子的集合。但是,自引力系统的总质量与总动量的情况仍然十分复杂。后面会看到,孤立系统的质量与动量\textbf{守恒},但这必须在对这些量下定义之后才能继续讨论。


\section{扩展阅读}
\label{sec7.5}
Geroch (1978) 详细讨论了曲率与物理学是怎样结合的。所有的广义相对论的高阶教材都深入探讨了守恒量。本章内容是不变的弯曲时空中量子场论的基础,参见Birrell and Davies (1984) and Wald (1994). 它是目前引力研究最活跃的前沿之一——广义相对论的量子化。本书并未涉及这些内容,希望从经典广义相对论出发研究引力量子化(另一种方法从弦论出发)的读者可以参考 Rovelli (2004) Bojowald (2005), and Thiemann (2007).


\section{习题}
\label{sec7.6}