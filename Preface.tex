\chapter*{第二版前言}
本书第一版发行23年至今,广义相对论研究领域已经发展成熟。广相已经在它坚实的数学基础之上发展了丰富的内容,其中一些在第一版出版的1985年时根本想象不到。因此广义相对论的研究已经从专业理论物理学家的边缘移动到了核心,越来越多的本科生希望在研究生阶段之前学习广义相对论的基础内容。

我的读者很有耐心,使用本书第一版的学生从中学到了广义相对论的数学基础,但它的一些应用内容严重过时,例如黑洞天体物理学、引力波探测以及宇宙学的相关内容。我希望增添、修订了的第二版可以finally bring the book back into balance and give
readers a consistent and unified introduction to modern research in classical gravitation

(未完成)


%\chapterimage{homura.png}
\chapter*{第一版前言}
本书来自我在1975-1980年的一学年广义相对论本科课程讲义,这次授课经历使我确信,广义相对论基础课不比本科必修课(例如电动力学和量子力学)困难太多。最近二十年以来,广义相对论的研究热点(主要由天文学驱动)大大增加,这不仅使得对理论的理解更加深刻、完善,而且启发了更加简单、更加有物理意义的理解方式。相对论已经是物理与天文的主流内容,任何理论物理学家都学习过这门课。在相对论出现的早期,它以困难而闻名(\textit{记者:‘Eddington教授,世界上只有三个人懂爱因斯坦的理论吗?’ Eddington:‘谁是第三个?’}),这种名声也许是推广广义相对论教学的主要障碍。本书的主要目的是以适合本科生的难度讲授广义相对论,学生将从中学习理解基础的物理概念和实验结论,解决一些基本问题,以及可以学习更高级的相对论文献。

为了这个目的,我努力实现两个互斥的标准:第一,对读者的背景知识要求最少;第二,避免稀释主题内容。与大多数入门教材不同,本书不需要读者学过相对论形式的电动力学、电磁波理论或者流体力学。必要的流体力学内容在专门的章节介绍。由于假设读者不熟悉电磁波,因此引力波需要较缓慢地讲述,需要从头讲述波动方程。下文给出了学习本书需要的背景知识。

第二个标准——避免稀释主题内容——是相当主观性的,很不容易描述。本书介绍了完整的微分几何理论,不止仅仅依靠曲面的类比,而且略去了对于广义相对论的入门阶段而言不必要的内容,例如非度规流形理论、李导数和纤维丛。\footnote{因此本书与我的另一本书\textit{Geometrical Methods of Mathematical Physics} (Cambridge University Press 1980b) 的风格不同,后者是本书的补充。} 本书介绍了完整的非线性场方程(而不仅仅是线性理论),但是只求解平面与球对称情形,此外还引用、研究Kerr解。引力波主要在线性近似下研究,但是在推导引力波能量和波发射体中的反应效应时更进一步。我尽力为相对论的每个领域都提供了足够的基础,从而使学生在进行深入研究时不需要从头学起。

本书第一部分(第1到第8章)以典型的顺序介绍了基本理论:狭义相对论回顾,狭义相对论下的张量分析与连续介质物理学,欧几里得空间和Minkowski空间中曲线坐标系下的张量微积分,弯曲流形的几何,弯曲时空中的物理学,最后是场方程。其余四章研究了一些在现代天体物理中十分重要的课题。引力辐射的章节比一般的入门教材更加详细,因为引力波的观测也许是天文学下一个十年最伟大的进展。球对称恒星模型的那一章除了通常的内容之外,还包含Buchdahl的一组常用的可压缩精确解。关于黑洞的很长的章节相当详细地研究了视界处的物理性质,以及Kruskal坐标系,然后探索了旋转(Kerr)黑洞,以及对Hawking效应(黑洞发射辐射的量子力学理论)的简单讨论。最后讨论宇宙学的一章推导了均匀各向同性度规,简单研究了宇宙学观测和宇宙演化的物理机制。附录部分包含了本书用到的线性代数内容的总结,以及部分习题的提示与解答。其它入门教材经常涉及,而本书并未特别讲述的内容是广义相对论的实验验证与其它替代的引力理论。 Points of contact with experiment are treated as they arise, but systematic discussions of tests now require whole books (Will 1981)\footnote{这本经典文献的修订了的第二版是Will(1993)。}. 广义相对论在今天比起十年二十年前更加被实验验证,因此我认为在目前的相对论入门教材讨论其它替代性理论是不合适的,就像在电磁学教材中讨论电磁学的替代理论那样。

本书假设读者已经学过:狭义相对论,包括Lorentz变换和相对论力学;欧几里得空间的向量分析;常微分方程和简单的偏微分方程;热力学和流体静力学;牛顿引力理论(最好了解简单的恒星结构,不知道也可以);基本的量子力学(知道光子是啥)。

我们的符号与惯例和Misner等人的著作\textit{Gravitation} (W.H.Freeman 1973)一致,此书可以作为学完本书后的进阶教材。我还借鉴了\textit{Gravitation}的物理观点与讲述方式,因为Thorne是我的老师,而且\textit{Gravitation}的影响力实在很大。但是本书是更简单的、面向更广泛人群的教材,因此两本书的写作风格与教学方式非常不同。

本书作为一学年的课程教材使用,也可以选择部分内容作为一学期课程。例如,为学过电磁波的学生开设的关于引力波和黑洞的一学期课程就可以跳过1-3章、第4章大部分、第7和10章。再例如为学过狭义相对论、流体力学的学生开设的关于恒星结构和宇宙学的一学期课程可以快速浏览前4章并跳过第9、11章。对于研究生课程,如果讲的够快,也能一学期讲完。

每一章后面都有习题,难度从残破级(例如补全正文省略的推导过程、运用新介绍的数学工具)到精英级(有些是正文内容的扩展)。一些习题需要可编程计算器或者计算机。做习题的重要性怎么强调都不为过,一些早期章节简单和中等的习题是教材知识点的实际应用,缺少这些经历则之后的内容难以理解。后期章节的中等、困难习题是对理解效果的检验。学生在学相对论的时候往往沉迷于有趣的概念框架,而把习题放到第二位。这种做法是错误而危险的:一个不能解出普通难度习题的人不可能真正理解理论。习题的数量比学生应该做的更多,每章有30到,教师应该在其中适当选择一些。Lightman等人的\textit{Problem Book in Relativity and Gravitation}, (Princeton University Press 1975) 是一本丰富的相对论习题集。

本书的写作要感谢许多人直接或间接的帮助。特别感谢我在Cardiff大学教授的本科生们,他们对这门课的热情,以及对早期课程讲义缺陷的包容,鼓励我完成了这本书。感谢 Suzanne Ball, Jane Owen, Margaret Vallender, Pranoat Priesmeyer, and Shirley Kemp 耐心打印手稿。
