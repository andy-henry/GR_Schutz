% 翻译:SI、Andy_Henry
% 校对:Andy_Henry、???
% 由于不明原因,本书作者的国籍在世图影印本中被标为英国。根据维基百科,作者的国籍应为美国。

\chapter*{第二版前言}
从本书第一版出版到现在修订版出版的23年里,广义相对论这一领域日趋成熟并蓬勃发展。在广义相对论坚实的数学基础之上,逐渐产生了大量实际应用,其中某些应用在第一版出版的1985年甚至是无法想象的。对广义相对论的研究也因此从专业理论物理教学的次要环节转变为核心环节,越来越多的学生希望在本科阶段至少能学到一些广义相对论的基本内容。

本书的读者们很有耐心,他们一直使用本书的第一版来学习广义相对论的数学基础,但是第一版中关于应用的内容已经严重过时了,例如黑洞天体物理学、探测引力波、探索宇宙等等。我希望对第二版所进行的全面修订,可以使本书不再过时%back into balance重回正常状态,这里意译。
,向读者们统一而连贯地介绍经典引力论的现代研究成果。

前八章几乎没有变化:纳入了供进一步阅读的参考文献,扩展了少数几节的内容。但整体上看,广相理论数学基础的几何方法经受住了时间的检验。与前八章不同,对于处理天体物理大舞台中广义相对论应用的后四章,则进行了大量更新与扩展,某些内容甚至完全重写了。

处理引力波的第9章,现在拓展了关于使用干涉测量装置(比如LIGO以及已计划建设的空间探测装置LISA)探测引力波的讨论,也加入了关于引力波可能来源的讨论,以及我们希望通过探测引力波来获得什么。这是一个快速变化中的领域,引力波的首次直接探测随时都会实现\footnote{LIGO已于2015年首次探测到引力波。——译者注}。第9章旨在提供一个用来理解这些探测活动内在原理的稳定框架。

第10章讨论了球对称结构的恒星为何能保持稳定,但也穿插了关于实际%此处real译法存疑
中子星的内容。中子星可以看作脉冲星,是可供探测的引力波的潜在来源。

处理黑洞的第11章,加入了关于证明黑洞存在的天体物理学证据的大量内容。这些证据证明,既存在恒星黑洞,也存在超大质量黑洞。令人惊讶的是,天文学家已经在大多数星系的中心发现了这两种大型黑洞的存在。对Hawking辐射的讨论也略有修正。

最后,处理宇宙学的第12章完全重写了。第一版中我从根本上忽视了宇宙常数,按照当时的看法假设宇宙的膨胀速度正在减慢,但当时还没有足够精确地测量出宇宙的膨胀速度。而现在我们深信,根据大量相互一致的观测结果,宇宙正在加速膨胀。这可能是今天理论物理学所面临的最大挑战,极大影响了粒子物理的基本理论,也极大影响了宇宙学问题。围绕以上观点,我组织了第12章的内容,阐述了一个纳入了宇宙学常数的,针对膨胀中宇宙的数学模型,然后详细讨论了天文学中如何度测量宇宙的膨胀速度,最后探究了大爆炸之后宇宙逐渐形成了的物质组成。宇宙膨胀,暗物质,暗能量,这些都在影响着今天的宇宙结构,甚至影响着我们的存在。通过这一章读者也许只能简短地初识天文学家从这些问题中获知了什么,但我希望这已足够鼓励读者去继续学习更多内容。

我在若干合适的章节纳入了更多习题,但从书中删去了习题解答。习题解答可以从本书的网站上获取。

本书的主题依然是经典的广义相对论,除了简要地讨论了Hanking辐射之外,不涉及量子引力。虽然量子引力是今天理论物理研究中最为活跃的领域之一,但现在仍然无法给想要学习引力量子化的学生指出一个清晰的方向。或许到第三版时就可能纳入处理引力量子化的一章了。

我要感谢帮助我完成了第二版写作的许多人。一些人向我大度地提供了第一版中印刷错误与内容错误的清单,这里尤其要提到Frode Appel、Robert D’Alessandro、J. A. D. Ewart、Steve Fulling、Toshi Futamase(二間瀬敏史)、Ted Jacobson、Gerald Quinlan、B. Sathyaprakash,当然书中剩余的其他错误我文责自负。我也要感谢剑桥大学出版社的编辑Rufus Neal、Simon Capelin、Lindsay Barnes,他们非常耐心并给予我鼓励。当然我还要深深地感谢我的妻子Sian,在我修订本书期间的每时每刻,她都展现了极大的耐心。

%\chapterimage{homura.png}
\chapter*{第一版前言}
我曾于1975年至1980年讲授过一门一学年的广义相对论本科课程,并逐渐形成了这本书。这次授课经历使我确信,本科生学习广义相对论时并不比学习其他本科水平的标准课程时(如电动力学与量子力学)存在显著困难。过去20年间,主要在天文学的驱动下,物理学界对广义相对论的研究兴趣迅速增加,这不仅使得我们更深入更完整地理解了广义相对论,也教会了我们以更简单更物理的方式理解广义相对论。相对论现已进入物理学与天文学的主流,缺少相对论学科训练的理论物理教学可以说是不完整的。在相对论出现的早期,其以令人敬畏的巨大难度而闻名(记者:“Eddington教授,世界上只有三个人理解Einstein理论,这是真的吗?”Eddington:“第三个人是谁?”)\footnote{Sir Arthur Stanley Eddington(亚瑟·斯坦利·爱丁顿爵士),英国天体物理学家、数学家,是第一个用英语宣讲相对论的科学家。——译者注},这也许是今天难以在理论物理教学中推广广义相对论的首要障碍。这本教材旨在以适合本科生水平展示广义相对论,使得学生可以理解广义相对论的基本物理概念与广义相对论实验的内在原理,能求解一些初等问题,并为学习更为高级的广义相对论教材做好准备。

为了实现这一目标,我努力去满足两个相互矛盾的原则:第一,假定读者只了解最低要求的预备知识;第二,避免稀释主题内容。不同于大多数入门教材,本书不假定学生学习过电磁学的显式相对论表述%原文为electromagnetism in its manifestly relativistic formulation
、电磁波理论和流体力学。必要的流体力学内容会在相关章节进行阐述。不假定读者熟悉电磁波的主要影响是不得不缓慢地引入引力波,从头研究波动方程。下文会完整列出阅读本书所需的预备知识。

第二条原则,避免稀释主题内容,非常主观而且很难描述。我设法在最大程度上引入微分几何,并不满足于只依赖与曲面的类比,%此处译法存疑
但我没有纳入基础教材中非必需的广义相对论主题,例如非度规流形理论、李导数和纤维丛。\footnote{因此这里的处理思想不同于我的另一本书\textit{Geometrical Methods of Mathematical Physics} (Cambridge University Press, 1980b),其可以作为本书的补充。}我在最大程度上引入了非线性场方程,而不仅仅是那些线性化理论,%此处译法存疑,those指代什么?
但只在平面与球对称两种情形下进行求解,此外还引用并考察了Kerr解。引力波主要在线性近似下进行研究,但是会稍微使用一点非线性%此处译法存疑,beyond指超出什么?是否指超出线性(使用非线性)?
来推导引力波的能量与引力波发射体中的反应效应。%此处译法存疑,且effect应按具体含义意译。
我设法为每个论题都打下足够多的基础,让学生能够在进行更为高等的研究时不必再从头学起。
%本段理应在熟悉图书相关内容后进行翻译,熟读相关内容后才能正确把握翻译。

本书的第一部分(截止到第8章)按照大多数教材的典型顺序引入了广义相对论:复习了狭义相对论,阐述了狭义相对论下的张量分析与连续介质物理,研究了Euclid空间与Minkowski空间中曲线坐标系下的张量分析,以及弯曲流形上的几何学,弯曲时空中的物理学,最后是场方程。其余四章我选择研究在现代天体物理中较为重要的几个论题。处理引力辐射的一章相比入门水平上的常规内容更为详细,这是因为对引力波的成功观测也许将成为未来十年天文学中意义最为重大的进展之一。处理球对称恒星的一章除了常规内容之外,还包含了一组有用的可压缩精确解(应归功于Buchdahl)。%译法存疑
关于黑洞的一章比较长,这一章相当详细地研究了视界处的物理性质,延伸讨论了Kruskal坐标系,然后探究了旋转黑洞(Kerr黑洞),以简单地讨论了Hawking效应(黑洞辐射的量子力学原理)来结束本章。最后一章处理宇宙学,推导出了均匀各向同性度规,并简单研究了宇宙观测与宇宙演化的物理学原理。一个附录总结了本教材所需要用到的线性代数知识,另一个附录包含了部分习题的提示与解答。其它教材长期以来涉及,而我决定本书不需特别突出的一个主题,是广义相对论及其它引力理论的实验验证。与实验验证有联系的重点内容会在用到时再做处理,但当前已经需要一整本书(Will 1981)\footnote{这本经典著作的修订第二版是Will (1993)。}才能系统讨论实验验证。对于广义相对论的正确性,今天的物理学界相比十几二十年前要更有信心,因此我认为在一本现代的基础教材中讨论其它引力理论是不大合适的,就像基础电磁学教材中不会讨论其他电磁理论那样。
%本段理应在熟悉图书相关内容后进行翻译,熟读相关内容后才能正确把握翻译。

本书假设学生已经学过:狭义相对论(包括Lorentz变换和相对论力学),Euclid空间的向量分析,常微分方程和简单的偏微分方程,热力学和流体静力学,牛顿引力理论(简单学过恒星星结构是有用的但非必须),足够的初等量子力学(知道光子是什么)。

本书的符号与约定与Misner等人的\textit{Gravitation} (W. H. Freeman, 1973)本质上相同,在阅读完本书后可以将\textit{Gravitation}作为后续教材之一。自然\textit{Gravitation}也影响了本书的物理观点视角与主题阐释,这一部分是因为Thorne曾是我的老师,一部分也是因为\textit{Gravitation}是一本很有影响力的教材。但由于我试图让更为广泛的读者理解将广义相对论这一主题,因此两本书的风格与教学方式是极为不同的。

关于本书的使用。虽然本书的设计是在一年的课程中按顺序学习整本书,但是可以将其压缩以适应半年的课程。比如说对于已经学过电磁波的学生,在半年中致力于引力波与黑洞的教学是合适的,这时要审慎地跳过第1~3章的部分内容和第4、7、10三章的大部分内容。再比如对于拥有狭义相对论与流体力学预备知识的学生,可以在半年中学完恒星结构与宇宙学,这时可以快速地讲授前四章并跳过第9、11两章。当然,讲授研究生课程时可以快得多,应当可以在半年内覆盖整本教材。

每一章后面都有一套习题,难度从零碎的习题(补全书中正文所缺少的步骤,使用新引入的数学方法)到很大程度上拓展了本书所讨论内容的高级问题。某些习题需要使用可编程计算器或计算机。选些题做的重要性再怎么强调都不为过。前半部分各章中的简单题和中等题给是基本练习,不做这些习题将很难理解后续各章。后半部分各章中的中等题和难题可以检验学生对广义相对论的理解。%个人猜测这里early chapters指前8章,later chapter指后4章。如果确实如此,则应当按此明确翻译。
一个非常常见的现象是,学生发现相对论的概念框架十分有趣,而将解题下放到第二位。这种将物理概念与解题相分离的做法是错误而危险的,不会求解一定难度习题的学生并不能真正理解理论概念。书中的习题数量一般会比学生应做的更多,有几章有30多到习题,教师必须仔细选择习题。Lightman等人的\textit{Problem Book in Relativity and Gravitation}, (Princeton University Press, 1975)也是一本丰富的习题来源。

我要感谢许多人对本书直接或间接的帮助。我想特别感谢我在Cardiff大学的本科生们,他们对这门学科的热情,以及对早期讲义中不足的忍耐,鼓励我将讲义变成了一本书。我当然还要感谢Suzanne Ball、Jane Owen、Margaret Vallender、Pranoat Priesmeyer和Shirley Kemp,他们耐心地录入了我陆续写作的手稿。
